\
\documentclass[11pt,a4paper]{article}
\usepackage{isabelle,isabellesym}
\usepackage{amsmath,amssymb,amsthm}
\usepackage[utf8]{inputenc}
\usepackage[T1]{fontenc}
\usepackage{hyperref}
\usepackage[strings]{underscore} % allow underscores in text

\title{Unified Field Theory\\[0.25em]\large A Machine-Checked Isabelle/HOL Development}
\author{Matthew Scherf}
\date{\today}

\begin{document}
\maketitle

\begin{abstract}
This entry presents a machine-checked formalization of a unified field theory grounded in non-dual ontology. The development consists of two main theories: \textit{The\_Unique\_Ontic\_Substrate}, which establishes the foundational metaphysical framework, and \textit{Unified\_Field\_Theory}, which interprets quantum fields, gauge structures, fundamental forces, and spacetime geometry within this framework.

The core axiomatization posits a unique ontic substrate ($\Omega$) from which all physical phenomena arise as presentation modes rather than independent entities. We formalize the principle of non-duality as inseparability: every phenomenon $p$ satisfies $\text{Inseparable}(p, \Omega)$. Building on this foundation, we develop formal treatments of quantum field excitations as substrate presentations, gauge symmetries as presentation indexing schemes, and the four fundamental forces (electromagnetic, weak, strong, gravitational) as unified presentation modes of the substrate.

The framework naturally accounts for quantum entanglement through substrate unity, with machine-verified proofs showing that entangled phenomena share presentations of the same substrate. Spacetime geometry emerges as a representational structure over phenomena rather than a fundamental arena, with curvature formalized as a frame-dependent quantity over presentations. Information-theoretic foundations are incorporated through abstract quantities satisfying holographic bounds, connecting fundamental constants to the geometry of presentation space.

We provide testable predictions that distinguish substrate-mediated correlations from local hidden variable theories, formalizing constraints that substrate correlations must be acausal and symmetric. The development includes implementation roadmaps specifying how abstract types map to operational criteria and how effective theories (Standard Model, General Relativity) emerge at appropriate scales.

All results are mechanically verified in Isabelle/HOL 2025, with proofs checked for consistency using Nitpick. The formalization demonstrates that a mathematically rigorous non-dual ontology can accommodate the full structure of modern physics while maintaining logical coherence and offering novel empirical predictions.
\end{abstract}

\tableofcontents
\bigskip

% Isabelle-generated document body:
\parindent 0pt \parskip 0.6ex
%
\begin{isabellebody}%
\setisabellecontext{The{\isacharunderscore}{\kern0pt}Unique{\isacharunderscore}{\kern0pt}Ontic{\isacharunderscore}{\kern0pt}Substrate}%
%
\isadelimtheory
%
\endisadelimtheory
%
\isatagtheory
\isakeywordONE{theory}\isamarkupfalse%
\ The{\isacharunderscore}{\kern0pt}Unique{\isacharunderscore}{\kern0pt}Ontic{\isacharunderscore}{\kern0pt}Substrate\isanewline
\ \ \isakeywordTWO{imports}\ Main\isanewline
\isakeywordTWO{begin}%
\endisatagtheory
{\isafoldtheory}%
%
\isadelimtheory
%
\endisadelimtheory
%
\isadelimdocument
%
\endisadelimdocument
%
\isatagdocument
%
\isamarkupsection{Core Ontology%
}
\isamarkuptrue%
%
\endisatagdocument
{\isafolddocument}%
%
\isadelimdocument
%
\endisadelimdocument
\isakeywordONE{typedecl}\isamarkupfalse%
\ E\ \ \isanewline
\isanewline
\isakeywordONE{consts}\isamarkupfalse%
\isanewline
\ \ Phenomenon\ {\isacharcolon}{\kern0pt}{\isacharcolon}{\kern0pt}\ {\isachardoublequoteopen}E\ {\isasymRightarrow}\ bool{\isachardoublequoteclose}\isanewline
\ \ Substrate\ \ {\isacharcolon}{\kern0pt}{\isacharcolon}{\kern0pt}\ {\isachardoublequoteopen}E\ {\isasymRightarrow}\ bool{\isachardoublequoteclose}\isanewline
\ \ Presents\ \ \ {\isacharcolon}{\kern0pt}{\isacharcolon}{\kern0pt}\ {\isachardoublequoteopen}E\ {\isasymRightarrow}\ E\ {\isasymRightarrow}\ bool{\isachardoublequoteclose}\ \ \ \isanewline
\ \ Inseparable\ {\isacharcolon}{\kern0pt}{\isacharcolon}{\kern0pt}\ {\isachardoublequoteopen}E\ {\isasymRightarrow}\ E\ {\isasymRightarrow}\ bool{\isachardoublequoteclose}\isanewline
\isanewline
\isakeywordONE{axiomatization}\isamarkupfalse%
\ \isakeywordTWO{where}\isanewline
\ \ A{\isadigit{1}}{\isacharunderscore}{\kern0pt}existence{\isacharcolon}{\kern0pt}\ \ \ \ \ {\isachardoublequoteopen}{\isasymexists}s{\isachardot}{\kern0pt}\ Substrate\ s{\isachardoublequoteclose}\ \isakeywordTWO{and}\isanewline
\ \ A{\isadigit{2}}{\isacharunderscore}{\kern0pt}uniqueness{\isacharcolon}{\kern0pt}\ \ \ \ {\isachardoublequoteopen}{\isasymforall}a\ b{\isachardot}{\kern0pt}\ Substrate\ a\ {\isasymlongrightarrow}\ Substrate\ b\ {\isasymlongrightarrow}\ a\ {\isacharequal}{\kern0pt}\ b{\isachardoublequoteclose}\ \isakeywordTWO{and}\isanewline
\ \ A{\isadigit{3}}{\isacharunderscore}{\kern0pt}exhaustivity{\isacharcolon}{\kern0pt}\ \ {\isachardoublequoteopen}{\isasymforall}x{\isachardot}{\kern0pt}\ Phenomenon\ x\ {\isasymor}\ Substrate\ x{\isachardoublequoteclose}\ \isakeywordTWO{and}\isanewline
\ \ A{\isadigit{4}}{\isacharunderscore}{\kern0pt}presentation{\isacharcolon}{\kern0pt}\ \ {\isachardoublequoteopen}{\isasymforall}p\ s{\isachardot}{\kern0pt}\ Phenomenon\ p\ {\isasymand}\ Substrate\ s\ {\isasymlongrightarrow}\ Presents\ p\ s{\isachardoublequoteclose}\ \isakeywordTWO{and}\isanewline
\ \ A{\isadigit{5}}{\isacharunderscore}{\kern0pt}insep{\isacharunderscore}{\kern0pt}def{\isacharcolon}{\kern0pt}\ \ \ \ \ {\isachardoublequoteopen}{\isasymforall}x\ y{\isachardot}{\kern0pt}\ Inseparable\ x\ y\ {\isasymlongleftrightarrow}\ {\isacharparenleft}{\kern0pt}{\isasymexists}s{\isachardot}{\kern0pt}\ Substrate\ s\ {\isasymand}\ Presents\ x\ s\ {\isasymand}\ y\ {\isacharequal}{\kern0pt}\ s{\isacharparenright}{\kern0pt}{\isachardoublequoteclose}\isanewline
\isanewline
\isakeywordONE{lemma}\isamarkupfalse%
\ unique{\isacharunderscore}{\kern0pt}substrate{\isacharcolon}{\kern0pt}\ {\isachardoublequoteopen}{\isasymexists}{\isacharbang}{\kern0pt}s{\isachardot}{\kern0pt}\ Substrate\ s{\isachardoublequoteclose}\isanewline
%
\isadelimproof
\ \ %
\endisadelimproof
%
\isatagproof
\isakeywordONE{using}\isamarkupfalse%
\ A{\isadigit{1}}{\isacharunderscore}{\kern0pt}existence\ A{\isadigit{2}}{\isacharunderscore}{\kern0pt}uniqueness\ \isakeywordONE{by}\isamarkupfalse%
\ {\isacharparenleft}{\kern0pt}metis{\isacharparenright}{\kern0pt}%
\endisatagproof
{\isafoldproof}%
%
\isadelimproof
\isanewline
%
\endisadelimproof
\isanewline
\isakeywordONE{definition}\isamarkupfalse%
\ TheSubstrate\ {\isacharcolon}{\kern0pt}{\isacharcolon}{\kern0pt}\ {\isachardoublequoteopen}E{\isachardoublequoteclose}\ \ {\isacharparenleft}{\kern0pt}{\isachardoublequoteopen}{\isasymOmega}{\isachardoublequoteclose}{\isacharparenright}{\kern0pt}\isanewline
\ \ \isakeywordTWO{where}\ {\isachardoublequoteopen}{\isasymOmega}\ {\isacharequal}{\kern0pt}\ {\isacharparenleft}{\kern0pt}SOME\ s{\isachardot}{\kern0pt}\ Substrate\ s{\isacharparenright}{\kern0pt}{\isachardoublequoteclose}\isanewline
\isanewline
\isakeywordONE{lemma}\isamarkupfalse%
\ substrate{\isacharunderscore}{\kern0pt}Omega{\isacharcolon}{\kern0pt}\ {\isachardoublequoteopen}Substrate\ {\isasymOmega}{\isachardoublequoteclose}\isanewline
%
\isadelimproof
\ \ %
\endisadelimproof
%
\isatagproof
\isakeywordONE{unfolding}\isamarkupfalse%
\ TheSubstrate{\isacharunderscore}{\kern0pt}def\ \isakeywordONE{using}\isamarkupfalse%
\ A{\isadigit{1}}{\isacharunderscore}{\kern0pt}existence\ someI{\isacharunderscore}{\kern0pt}ex\ \isakeywordONE{by}\isamarkupfalse%
\ metis%
\endisatagproof
{\isafoldproof}%
%
\isadelimproof
\isanewline
%
\endisadelimproof
\isanewline
\isakeywordONE{lemma}\isamarkupfalse%
\ only{\isacharunderscore}{\kern0pt}substrate{\isacharunderscore}{\kern0pt}is{\isacharunderscore}{\kern0pt}Omega{\isacharcolon}{\kern0pt}\ {\isachardoublequoteopen}Substrate\ s\ {\isasymLongrightarrow}\ s\ {\isacharequal}{\kern0pt}\ {\isasymOmega}{\isachardoublequoteclose}\isanewline
%
\isadelimproof
\ \ %
\endisadelimproof
%
\isatagproof
\isakeywordONE{using}\isamarkupfalse%
\ substrate{\isacharunderscore}{\kern0pt}Omega\ A{\isadigit{2}}{\isacharunderscore}{\kern0pt}uniqueness\ \isakeywordONE{by}\isamarkupfalse%
\ blast%
\endisatagproof
{\isafoldproof}%
%
\isadelimproof
\isanewline
%
\endisadelimproof
\isanewline
\isakeywordONE{lemma}\isamarkupfalse%
\ consistency{\isacharunderscore}{\kern0pt}witness{\isacharcolon}{\kern0pt}\ True%
\isadelimproof
\ %
\endisadelimproof
%
\isatagproof
\isakeywordONE{by}\isamarkupfalse%
\ simp%
\endisatagproof
{\isafoldproof}%
%
\isadelimproof
%
\endisadelimproof
%
\isadelimdocument
%
\endisadelimdocument
%
\isatagdocument
%
\isamarkupsection{Non-Duality%
}
\isamarkuptrue%
%
\endisatagdocument
{\isafolddocument}%
%
\isadelimdocument
%
\endisadelimdocument
\isakeywordONE{theorem}\isamarkupfalse%
\ Nonduality{\isacharcolon}{\kern0pt}\isanewline
\ \ {\isachardoublequoteopen}{\isasymforall}p{\isachardot}{\kern0pt}\ Phenomenon\ p\ {\isasymlongrightarrow}\ Inseparable\ p\ {\isasymOmega}{\isachardoublequoteclose}\isanewline
%
\isadelimproof
%
\endisadelimproof
%
\isatagproof
\isakeywordONE{proof}\isamarkupfalse%
\ {\isacharparenleft}{\kern0pt}intro\ allI\ impI{\isacharparenright}{\kern0pt}\isanewline
\ \ \isakeywordTHREE{fix}\isamarkupfalse%
\ p\ \isakeywordTHREE{assume}\isamarkupfalse%
\ P{\isacharcolon}{\kern0pt}\ {\isachardoublequoteopen}Phenomenon\ p{\isachardoublequoteclose}\isanewline
\ \ \isakeywordONE{from}\isamarkupfalse%
\ P\ substrate{\isacharunderscore}{\kern0pt}Omega\ A{\isadigit{4}}{\isacharunderscore}{\kern0pt}presentation\ \isakeywordONE{have}\isamarkupfalse%
\ {\isachardoublequoteopen}Presents\ p\ {\isasymOmega}{\isachardoublequoteclose}\ \isakeywordONE{by}\isamarkupfalse%
\ blast\isanewline
\ \ \isakeywordONE{hence}\isamarkupfalse%
\ {\isachardoublequoteopen}{\isasymexists}s{\isachardot}{\kern0pt}\ Substrate\ s\ {\isasymand}\ Presents\ p\ s\ {\isasymand}\ {\isasymOmega}\ {\isacharequal}{\kern0pt}\ s{\isachardoublequoteclose}\isanewline
\ \ \ \ \isakeywordONE{using}\isamarkupfalse%
\ substrate{\isacharunderscore}{\kern0pt}Omega\ \isakeywordONE{by}\isamarkupfalse%
\ blast\isanewline
\ \ \isakeywordTHREE{thus}\isamarkupfalse%
\ {\isachardoublequoteopen}Inseparable\ p\ {\isasymOmega}{\isachardoublequoteclose}\isanewline
\ \ \ \ \isakeywordONE{using}\isamarkupfalse%
\ A{\isadigit{5}}{\isacharunderscore}{\kern0pt}insep{\isacharunderscore}{\kern0pt}def\ \isakeywordONE{by}\isamarkupfalse%
\ blast\isanewline
\isakeywordONE{qed}\isamarkupfalse%
%
\endisatagproof
{\isafoldproof}%
%
\isadelimproof
%
\endisadelimproof
%
\isadelimdocument
%
\endisadelimdocument
%
\isatagdocument
%
\isamarkupsection{Causality (Phenomenon-Level)%
}
\isamarkuptrue%
%
\endisatagdocument
{\isafolddocument}%
%
\isadelimdocument
%
\endisadelimdocument
\isakeywordONE{consts}\isamarkupfalse%
\ CausallyPrecedes\ {\isacharcolon}{\kern0pt}{\isacharcolon}{\kern0pt}\ {\isachardoublequoteopen}E\ {\isasymRightarrow}\ E\ {\isasymRightarrow}\ bool{\isachardoublequoteclose}\ \ \ \isanewline
\isanewline
\isakeywordONE{axiomatization}\isamarkupfalse%
\ \isakeywordTWO{where}\isanewline
\ \ C{\isadigit{1}}{\isacharunderscore}{\kern0pt}only{\isacharunderscore}{\kern0pt}phenomena{\isacharcolon}{\kern0pt}\ {\isachardoublequoteopen}{\isasymforall}x\ y{\isachardot}{\kern0pt}\ CausallyPrecedes\ x\ y\ {\isasymlongrightarrow}\ Phenomenon\ x\ {\isasymand}\ Phenomenon\ y{\isachardoublequoteclose}\ \isakeywordTWO{and}\isanewline
\ \ C{\isadigit{2}}{\isacharunderscore}{\kern0pt}irreflexive{\isacharcolon}{\kern0pt}\ \ \ \ {\isachardoublequoteopen}{\isasymforall}x{\isachardot}{\kern0pt}\ Phenomenon\ x\ {\isasymlongrightarrow}\ {\isasymnot}\ CausallyPrecedes\ x\ x{\isachardoublequoteclose}\ \isakeywordTWO{and}\isanewline
\ \ C{\isadigit{3}}{\isacharunderscore}{\kern0pt}transitive{\isacharcolon}{\kern0pt}\ \ \ \ \ {\isachardoublequoteopen}{\isasymforall}x\ y\ z{\isachardot}{\kern0pt}\ CausallyPrecedes\ x\ y\ {\isasymand}\ CausallyPrecedes\ y\ z\ {\isasymlongrightarrow}\ CausallyPrecedes\ x\ z{\isachardoublequoteclose}\isanewline
\isanewline
\isakeywordONE{lemma}\isamarkupfalse%
\ Causal{\isacharunderscore}{\kern0pt}left{\isacharunderscore}{\kern0pt}NotTwo{\isacharcolon}{\kern0pt}\isanewline
\ \ \isakeywordTWO{assumes}\ {\isachardoublequoteopen}CausallyPrecedes\ x\ y{\isachardoublequoteclose}\ \isakeywordTWO{shows}\ {\isachardoublequoteopen}Inseparable\ x\ {\isasymOmega}{\isachardoublequoteclose}\isanewline
%
\isadelimproof
\ \ %
\endisadelimproof
%
\isatagproof
\isakeywordONE{using}\isamarkupfalse%
\ assms\ C{\isadigit{1}}{\isacharunderscore}{\kern0pt}only{\isacharunderscore}{\kern0pt}phenomena\ Nonduality\ \isakeywordONE{by}\isamarkupfalse%
\ blast%
\endisatagproof
{\isafoldproof}%
%
\isadelimproof
\isanewline
%
\endisadelimproof
\isanewline
\isakeywordONE{lemma}\isamarkupfalse%
\ Causal{\isacharunderscore}{\kern0pt}right{\isacharunderscore}{\kern0pt}NotTwo{\isacharcolon}{\kern0pt}\isanewline
\ \ \isakeywordTWO{assumes}\ {\isachardoublequoteopen}CausallyPrecedes\ x\ y{\isachardoublequoteclose}\ \isakeywordTWO{shows}\ {\isachardoublequoteopen}Inseparable\ y\ {\isasymOmega}{\isachardoublequoteclose}\isanewline
%
\isadelimproof
\ \ %
\endisadelimproof
%
\isatagproof
\isakeywordONE{using}\isamarkupfalse%
\ assms\ C{\isadigit{1}}{\isacharunderscore}{\kern0pt}only{\isacharunderscore}{\kern0pt}phenomena\ Nonduality\ \isakeywordONE{by}\isamarkupfalse%
\ blast%
\endisatagproof
{\isafoldproof}%
%
\isadelimproof
%
\endisadelimproof
%
\isadelimdocument
%
\endisadelimdocument
%
\isatagdocument
%
\isamarkupsection{Spacetime as Representation (Coordinates only for Phenomena)%
}
\isamarkuptrue%
%
\endisatagdocument
{\isafolddocument}%
%
\isadelimdocument
%
\endisadelimdocument
\isakeywordONE{typedecl}\isamarkupfalse%
\ Frame\isanewline
\isakeywordONE{typedecl}\isamarkupfalse%
\ R{\isadigit{4}}\ \ \ \ \ \isanewline
\isanewline
\isakeywordONE{consts}\isamarkupfalse%
\isanewline
\ \ coord\ \ \ \ {\isacharcolon}{\kern0pt}{\isacharcolon}{\kern0pt}\ {\isachardoublequoteopen}Frame\ {\isasymRightarrow}\ E\ {\isasymRightarrow}\ R{\isadigit{4}}\ option{\isachardoublequoteclose}\isanewline
\ \ GaugeRel\ {\isacharcolon}{\kern0pt}{\isacharcolon}{\kern0pt}\ {\isachardoublequoteopen}Frame\ {\isasymRightarrow}\ Frame\ {\isasymRightarrow}\ bool{\isachardoublequoteclose}\isanewline
\isanewline
\isakeywordONE{axiomatization}\isamarkupfalse%
\ \isakeywordTWO{where}\isanewline
\ \ S{\isadigit{1}}{\isacharunderscore}{\kern0pt}coords{\isacharunderscore}{\kern0pt}only{\isacharunderscore}{\kern0pt}for{\isacharunderscore}{\kern0pt}phenomena{\isacharcolon}{\kern0pt}\isanewline
\ \ \ \ {\isachardoublequoteopen}{\isasymforall}f\ x\ r{\isachardot}{\kern0pt}\ coord\ f\ x\ {\isacharequal}{\kern0pt}\ Some\ r\ {\isasymlongrightarrow}\ Phenomenon\ x{\isachardoublequoteclose}\ \isakeywordTWO{and}\isanewline
\ \ S{\isadigit{2}}{\isacharunderscore}{\kern0pt}gauge{\isacharunderscore}{\kern0pt}invariance{\isacharunderscore}{\kern0pt}definedness{\isacharcolon}{\kern0pt}\isanewline
\ \ \ \ {\isachardoublequoteopen}{\isasymforall}f\ g\ x{\isachardot}{\kern0pt}\ GaugeRel\ f\ g\ {\isasymlongrightarrow}\ {\isacharparenleft}{\kern0pt}coord\ f\ x\ {\isacharequal}{\kern0pt}\ None\ {\isasymlongleftrightarrow}\ coord\ g\ x\ {\isacharequal}{\kern0pt}\ None{\isacharparenright}{\kern0pt}{\isachardoublequoteclose}\isanewline
\isanewline
\isakeywordONE{lemma}\isamarkupfalse%
\ Spacetime{\isacharunderscore}{\kern0pt}unreality{\isacharcolon}{\kern0pt}\isanewline
\ \ \isakeywordTWO{assumes}\ {\isachardoublequoteopen}coord\ f\ x\ {\isasymnoteq}\ None{\isachardoublequoteclose}\isanewline
\ \ \isakeywordTWO{shows}\ {\isachardoublequoteopen}Inseparable\ x\ {\isasymOmega}{\isachardoublequoteclose}\isanewline
%
\isadelimproof
%
\endisadelimproof
%
\isatagproof
\isakeywordONE{proof}\isamarkupfalse%
\ {\isacharminus}{\kern0pt}\isanewline
\ \ \isakeywordONE{from}\isamarkupfalse%
\ assms\ \isakeywordTHREE{obtain}\isamarkupfalse%
\ r\ \isakeywordTWO{where}\ {\isachardoublequoteopen}coord\ f\ x\ {\isacharequal}{\kern0pt}\ Some\ r{\isachardoublequoteclose}\ \isakeywordONE{by}\isamarkupfalse%
\ {\isacharparenleft}{\kern0pt}cases\ {\isachardoublequoteopen}coord\ f\ x{\isachardoublequoteclose}{\isacharparenright}{\kern0pt}\ auto\isanewline
\ \ \isakeywordONE{hence}\isamarkupfalse%
\ {\isachardoublequoteopen}Phenomenon\ x{\isachardoublequoteclose}\ \isakeywordONE{using}\isamarkupfalse%
\ S{\isadigit{1}}{\isacharunderscore}{\kern0pt}coords{\isacharunderscore}{\kern0pt}only{\isacharunderscore}{\kern0pt}for{\isacharunderscore}{\kern0pt}phenomena\ \isakeywordONE{by}\isamarkupfalse%
\ blast\isanewline
\ \ \isakeywordTHREE{thus}\isamarkupfalse%
\ {\isachardoublequoteopen}Inseparable\ x\ {\isasymOmega}{\isachardoublequoteclose}\ \isakeywordONE{using}\isamarkupfalse%
\ Nonduality\ \isakeywordONE{by}\isamarkupfalse%
\ blast\isanewline
\isakeywordONE{qed}\isamarkupfalse%
%
\endisatagproof
{\isafoldproof}%
%
\isadelimproof
%
\endisadelimproof
%
\isadelimdocument
%
\endisadelimdocument
%
\isatagdocument
%
\isamarkupsection{Emptiness: No Intrinsic Essence of Phenomena%
}
\isamarkuptrue%
%
\endisatagdocument
{\isafolddocument}%
%
\isadelimdocument
%
\endisadelimdocument
\isakeywordONE{consts}\isamarkupfalse%
\ Essence\ {\isacharcolon}{\kern0pt}{\isacharcolon}{\kern0pt}\ {\isachardoublequoteopen}E\ {\isasymRightarrow}\ bool{\isachardoublequoteclose}\isanewline
\isanewline
\isakeywordONE{axiomatization}\isamarkupfalse%
\ \isakeywordTWO{where}\isanewline
\ \ Emptiness{\isacharunderscore}{\kern0pt}of{\isacharunderscore}{\kern0pt}Phenomena{\isacharcolon}{\kern0pt}\ {\isachardoublequoteopen}{\isasymforall}x{\isachardot}{\kern0pt}\ Phenomenon\ x\ {\isasymlongrightarrow}\ {\isasymnot}\ Essence\ x{\isachardoublequoteclose}%
\isadelimdocument
%
\endisadelimdocument
%
\isatagdocument
%
\isamarkupsection{Endogenous / Dependent Arising%
}
\isamarkuptrue%
%
\endisatagdocument
{\isafolddocument}%
%
\isadelimdocument
%
\endisadelimdocument
\isakeywordONE{consts}\isamarkupfalse%
\ ArisesFrom\ {\isacharcolon}{\kern0pt}{\isacharcolon}{\kern0pt}\ {\isachardoublequoteopen}E\ {\isasymRightarrow}\ E\ {\isasymRightarrow}\ bool{\isachardoublequoteclose}\ \ \ \isanewline
\isanewline
\isakeywordONE{axiomatization}\isamarkupfalse%
\ \isakeywordTWO{where}\isanewline
\ \ AF{\isacharunderscore}{\kern0pt}only{\isacharunderscore}{\kern0pt}pheno{\isacharcolon}{\kern0pt}\ \ \ {\isachardoublequoteopen}{\isasymforall}p\ q{\isachardot}{\kern0pt}\ ArisesFrom\ p\ q\ {\isasymlongrightarrow}\ Phenomenon\ p\ {\isasymand}\ Phenomenon\ q{\isachardoublequoteclose}\ \isakeywordTWO{and}\isanewline
\ \ AF{\isacharunderscore}{\kern0pt}endogenous{\isacharcolon}{\kern0pt}\ \ \ {\isachardoublequoteopen}{\isasymforall}p\ q{\isachardot}{\kern0pt}\ ArisesFrom\ p\ q\ {\isasymlongrightarrow}\ {\isacharparenleft}{\kern0pt}{\isasymexists}s{\isachardot}{\kern0pt}\ Substrate\ s\ {\isasymand}\ Presents\ p\ s\ {\isasymand}\ Presents\ q\ s{\isacharparenright}{\kern0pt}{\isachardoublequoteclose}\ \isakeywordTWO{and}\isanewline
\ \ AF{\isacharunderscore}{\kern0pt}no{\isacharunderscore}{\kern0pt}exogenous{\isacharcolon}{\kern0pt}\ {\isachardoublequoteopen}{\isasymforall}p\ q{\isachardot}{\kern0pt}\ ArisesFrom\ p\ q\ {\isasymlongrightarrow}\ {\isasymnot}\ {\isacharparenleft}{\kern0pt}{\isasymexists}z{\isachardot}{\kern0pt}\ {\isasymnot}\ Phenomenon\ z\ {\isasymand}\ {\isasymnot}\ Substrate\ z{\isacharparenright}{\kern0pt}{\isachardoublequoteclose}%
\isadelimdocument
%
\endisadelimdocument
%
\isatagdocument
%
\isamarkupsection{Non-Appropriation (Ownership is Conventional)%
}
\isamarkuptrue%
%
\endisatagdocument
{\isafolddocument}%
%
\isadelimdocument
%
\endisadelimdocument
\isakeywordONE{typedecl}\isamarkupfalse%
\ Agent\isanewline
\isakeywordONE{consts}\isamarkupfalse%
\ Owns\ {\isacharcolon}{\kern0pt}{\isacharcolon}{\kern0pt}\ {\isachardoublequoteopen}Agent\ {\isasymRightarrow}\ E\ {\isasymRightarrow}\ bool{\isachardoublequoteclose}\isanewline
\isakeywordONE{consts}\isamarkupfalse%
\ ValidConv\ {\isacharcolon}{\kern0pt}{\isacharcolon}{\kern0pt}\ {\isachardoublequoteopen}E\ {\isasymRightarrow}\ bool{\isachardoublequoteclose}\isanewline
\isanewline
\isakeywordONE{axiomatization}\isamarkupfalse%
\ \isakeywordTWO{where}\isanewline
\ \ Ownership{\isacharunderscore}{\kern0pt}is{\isacharunderscore}{\kern0pt}conventional{\isacharcolon}{\kern0pt}\isanewline
\ \ \ \ {\isachardoublequoteopen}{\isasymforall}a\ p{\isachardot}{\kern0pt}\ Owns\ a\ p\ {\isasymlongrightarrow}\ Phenomenon\ p\ {\isasymand}\ ValidConv\ p{\isachardoublequoteclose}\ \isakeywordTWO{and}\isanewline
\ \ No{\isacharunderscore}{\kern0pt}ontic{\isacharunderscore}{\kern0pt}ownership{\isacharcolon}{\kern0pt}\isanewline
\ \ \ \ {\isachardoublequoteopen}{\isasymforall}a\ p{\isachardot}{\kern0pt}\ Owns\ a\ p\ {\isasymlongrightarrow}\ Inseparable\ p\ {\isasymOmega}\ {\isasymand}\ {\isasymnot}\ Essence\ p{\isachardoublequoteclose}%
\isadelimdocument
%
\endisadelimdocument
%
\isatagdocument
%
\isamarkupsection{Symmetry / Gauge on Phenomena%
}
\isamarkuptrue%
%
\endisatagdocument
{\isafolddocument}%
%
\isadelimdocument
%
\endisadelimdocument
\isakeywordONE{typedecl}\isamarkupfalse%
\ G\isanewline
\isakeywordONE{consts}\isamarkupfalse%
\ act\ {\isacharcolon}{\kern0pt}{\isacharcolon}{\kern0pt}\ {\isachardoublequoteopen}G\ {\isasymRightarrow}\ E\ {\isasymRightarrow}\ E{\isachardoublequoteclose}\ \ \ \isanewline
\isanewline
\isakeywordONE{axiomatization}\isamarkupfalse%
\ \isakeywordTWO{where}\isanewline
\ \ Act{\isacharunderscore}{\kern0pt}closed{\isacharcolon}{\kern0pt}\ \ \ \ \ \ \ \ \ \ \ \ {\isachardoublequoteopen}{\isasymforall}g\ x{\isachardot}{\kern0pt}\ Phenomenon\ x\ {\isasymlongrightarrow}\ Phenomenon\ {\isacharparenleft}{\kern0pt}act\ g\ x{\isacharparenright}{\kern0pt}{\isachardoublequoteclose}\ \isakeywordTWO{and}\isanewline
\ \ Act{\isacharunderscore}{\kern0pt}pres{\isacharunderscore}{\kern0pt}presentation{\isacharcolon}{\kern0pt}\ {\isachardoublequoteopen}{\isasymforall}g\ x{\isachardot}{\kern0pt}\ Presents\ x\ {\isasymOmega}\ {\isasymlongrightarrow}\ Presents\ {\isacharparenleft}{\kern0pt}act\ g\ x{\isacharparenright}{\kern0pt}\ {\isasymOmega}{\isachardoublequoteclose}\isanewline
\isanewline
\isakeywordONE{lemma}\isamarkupfalse%
\ Symmetry{\isacharunderscore}{\kern0pt}preserves{\isacharunderscore}{\kern0pt}NotTwo{\isacharcolon}{\kern0pt}\isanewline
\ \ \isakeywordTWO{assumes}\ {\isachardoublequoteopen}Phenomenon\ x{\isachardoublequoteclose}\isanewline
\ \ \isakeywordTWO{shows}\ {\isachardoublequoteopen}Inseparable\ {\isacharparenleft}{\kern0pt}act\ g\ x{\isacharparenright}{\kern0pt}\ {\isasymOmega}{\isachardoublequoteclose}\isanewline
%
\isadelimproof
\ \ %
\endisadelimproof
%
\isatagproof
\isakeywordONE{using}\isamarkupfalse%
\ assms\ Act{\isacharunderscore}{\kern0pt}closed\ Act{\isacharunderscore}{\kern0pt}pres{\isacharunderscore}{\kern0pt}presentation\ A{\isadigit{5}}{\isacharunderscore}{\kern0pt}insep{\isacharunderscore}{\kern0pt}def\ substrate{\isacharunderscore}{\kern0pt}Omega\ Nonduality\isanewline
\ \ \isakeywordONE{by}\isamarkupfalse%
\ {\isacharparenleft}{\kern0pt}metis{\isacharparenright}{\kern0pt}%
\endisatagproof
{\isafoldproof}%
%
\isadelimproof
%
\endisadelimproof
%
\isadelimdocument
%
\endisadelimdocument
%
\isatagdocument
%
\isamarkupsection{Concepts / Annotations%
}
\isamarkuptrue%
%
\endisatagdocument
{\isafolddocument}%
%
\isadelimdocument
%
\endisadelimdocument
\isakeywordONE{typedecl}\isamarkupfalse%
\ Concept\isanewline
\isakeywordONE{consts}\isamarkupfalse%
\ Applies\ {\isacharcolon}{\kern0pt}{\isacharcolon}{\kern0pt}\ {\isachardoublequoteopen}Concept\ {\isasymRightarrow}\ E\ {\isasymRightarrow}\ bool{\isachardoublequoteclose}\isanewline
\isanewline
\isakeywordONE{axiomatization}\isamarkupfalse%
\ \isakeywordTWO{where}\isanewline
\ \ Concepts{\isacharunderscore}{\kern0pt}are{\isacharunderscore}{\kern0pt}annotations{\isacharcolon}{\kern0pt}\isanewline
\ \ \ \ {\isachardoublequoteopen}{\isasymforall}c\ x{\isachardot}{\kern0pt}\ Applies\ c\ x\ {\isasymlongrightarrow}\ Phenomenon\ x{\isachardoublequoteclose}\isanewline
\isanewline
\isakeywordONE{lemma}\isamarkupfalse%
\ Concepts{\isacharunderscore}{\kern0pt}don{\isacharprime}{\kern0pt}t{\isacharunderscore}{\kern0pt}reify{\isacharcolon}{\kern0pt}\isanewline
\ \ \isakeywordTWO{assumes}\ {\isachardoublequoteopen}Applies\ c\ x{\isachardoublequoteclose}\ \isakeywordTWO{shows}\ {\isachardoublequoteopen}Inseparable\ x\ {\isasymOmega}{\isachardoublequoteclose}\isanewline
%
\isadelimproof
\ \ %
\endisadelimproof
%
\isatagproof
\isakeywordONE{using}\isamarkupfalse%
\ assms\ Concepts{\isacharunderscore}{\kern0pt}are{\isacharunderscore}{\kern0pt}annotations\ Nonduality\ \isakeywordONE{by}\isamarkupfalse%
\ blast%
\endisatagproof
{\isafoldproof}%
%
\isadelimproof
%
\endisadelimproof
%
\isadelimdocument
%
\endisadelimdocument
%
\isatagdocument
%
\isamarkupsection{Quantities for Information and Time%
}
\isamarkuptrue%
%
\endisatagdocument
{\isafolddocument}%
%
\isadelimdocument
%
\endisadelimdocument
\isakeywordONE{typedecl}\isamarkupfalse%
\ Q%
\isadelimdocument
%
\endisadelimdocument
%
\isatagdocument
%
\isamarkupsection{Information Layer (Abstract Nonnegativity)%
}
\isamarkuptrue%
%
\endisatagdocument
{\isafolddocument}%
%
\isadelimdocument
%
\endisadelimdocument
\isakeywordONE{consts}\isamarkupfalse%
\isanewline
\ \ Info\ \ \ {\isacharcolon}{\kern0pt}{\isacharcolon}{\kern0pt}\ {\isachardoublequoteopen}E\ {\isasymRightarrow}\ Q{\isachardoublequoteclose}\isanewline
\ \ Nonneg\ {\isacharcolon}{\kern0pt}{\isacharcolon}{\kern0pt}\ {\isachardoublequoteopen}Q\ {\isasymRightarrow}\ bool{\isachardoublequoteclose}\isanewline
\isanewline
\isakeywordONE{axiomatization}\isamarkupfalse%
\ \isakeywordTWO{where}\isanewline
\ \ Info{\isacharunderscore}{\kern0pt}nonneg{\isacharcolon}{\kern0pt}\ {\isachardoublequoteopen}{\isasymforall}x{\isachardot}{\kern0pt}\ Phenomenon\ x\ {\isasymlongrightarrow}\ Nonneg\ {\isacharparenleft}{\kern0pt}Info\ x{\isacharparenright}{\kern0pt}{\isachardoublequoteclose}\isanewline
\isanewline
\isakeywordONE{lemma}\isamarkupfalse%
\ Info{\isacharunderscore}{\kern0pt}nonreifying{\isacharcolon}{\kern0pt}\isanewline
\ \ \isakeywordTWO{assumes}\ {\isachardoublequoteopen}Phenomenon\ x{\isachardoublequoteclose}\ \isakeywordTWO{shows}\ {\isachardoublequoteopen}Inseparable\ x\ {\isasymOmega}{\isachardoublequoteclose}\isanewline
%
\isadelimproof
\ \ %
\endisadelimproof
%
\isatagproof
\isakeywordONE{using}\isamarkupfalse%
\ assms\ Nonduality\ \isakeywordONE{by}\isamarkupfalse%
\ blast%
\endisatagproof
{\isafoldproof}%
%
\isadelimproof
%
\endisadelimproof
%
\isadelimdocument
%
\endisadelimdocument
%
\isatagdocument
%
\isamarkupsection{Emergent Time (Abstract Strict Order on Q)%
}
\isamarkuptrue%
%
\endisatagdocument
{\isafolddocument}%
%
\isadelimdocument
%
\endisadelimdocument
\isakeywordONE{consts}\isamarkupfalse%
\isanewline
\ \ T\ \ {\isacharcolon}{\kern0pt}{\isacharcolon}{\kern0pt}\ {\isachardoublequoteopen}E\ {\isasymRightarrow}\ Q{\isachardoublequoteclose}\ \ \ \ \ \ \ \ \ \ \ \isanewline
\ \ LT\ {\isacharcolon}{\kern0pt}{\isacharcolon}{\kern0pt}\ {\isachardoublequoteopen}Q\ {\isasymRightarrow}\ Q\ {\isasymRightarrow}\ bool{\isachardoublequoteclose}\ \ \ \isanewline
\isanewline
\isakeywordONE{axiomatization}\isamarkupfalse%
\ \isakeywordTWO{where}\isanewline
\ \ LT{\isacharunderscore}{\kern0pt}irrefl{\isacharcolon}{\kern0pt}\ \ \ \ \ {\isachardoublequoteopen}{\isasymforall}q{\isachardot}{\kern0pt}\ {\isasymnot}\ LT\ q\ q{\isachardoublequoteclose}\ \isakeywordTWO{and}\isanewline
\ \ LT{\isacharunderscore}{\kern0pt}trans{\isacharcolon}{\kern0pt}\ \ \ \ \ \ {\isachardoublequoteopen}{\isasymforall}a\ b\ c{\isachardot}{\kern0pt}\ LT\ a\ b\ {\isasymand}\ LT\ b\ c\ {\isasymlongrightarrow}\ LT\ a\ c{\isachardoublequoteclose}\ \isakeywordTWO{and}\isanewline
\ \ Time{\isacharunderscore}{\kern0pt}monotone{\isacharcolon}{\kern0pt}\ {\isachardoublequoteopen}{\isasymforall}x\ y{\isachardot}{\kern0pt}\ CausallyPrecedes\ x\ y\ {\isasymlongrightarrow}\ LT\ {\isacharparenleft}{\kern0pt}T\ x{\isacharparenright}{\kern0pt}\ {\isacharparenleft}{\kern0pt}T\ y{\isacharparenright}{\kern0pt}{\isachardoublequoteclose}\isanewline
\isanewline
\isakeywordONE{lemma}\isamarkupfalse%
\ Time{\isacharunderscore}{\kern0pt}emergent{\isacharunderscore}{\kern0pt}NotTwo{\isacharcolon}{\kern0pt}\isanewline
\ \ \isakeywordTWO{assumes}\ {\isachardoublequoteopen}Phenomenon\ x{\isachardoublequoteclose}\ \isakeywordTWO{shows}\ {\isachardoublequoteopen}Inseparable\ x\ {\isasymOmega}{\isachardoublequoteclose}\isanewline
%
\isadelimproof
\ \ %
\endisadelimproof
%
\isatagproof
\isakeywordONE{using}\isamarkupfalse%
\ assms\ Nonduality\ \isakeywordONE{by}\isamarkupfalse%
\ blast%
\endisatagproof
{\isafoldproof}%
%
\isadelimproof
%
\endisadelimproof
%
\isadelimdocument
%
\endisadelimdocument
%
\isatagdocument
%
\isamarkupsection{Two-Levels Coherence%
}
\isamarkuptrue%
%
\endisatagdocument
{\isafolddocument}%
%
\isadelimdocument
%
\endisadelimdocument
\isakeywordONE{consts}\isamarkupfalse%
\ Coherent\ {\isacharcolon}{\kern0pt}{\isacharcolon}{\kern0pt}\ {\isachardoublequoteopen}E\ {\isasymRightarrow}\ bool{\isachardoublequoteclose}\isanewline
\isanewline
\isakeywordONE{axiomatization}\isamarkupfalse%
\ \isakeywordTWO{where}\isanewline
\ \ Conventional{\isacharunderscore}{\kern0pt}is{\isacharunderscore}{\kern0pt}model{\isacharunderscore}{\kern0pt}relative{\isacharcolon}{\kern0pt}\ {\isachardoublequoteopen}{\isasymforall}x{\isachardot}{\kern0pt}\ ValidConv\ x\ {\isasymlongrightarrow}\ Phenomenon\ x{\isachardoublequoteclose}\ \isakeywordTWO{and}\isanewline
\ \ Ultimate{\isacharunderscore}{\kern0pt}coherence{\isacharcolon}{\kern0pt}\ \ \ \ \ \ \ \ \ \ \ \ \ {\isachardoublequoteopen}Coherent\ {\isasymOmega}{\isachardoublequoteclose}%
\isadelimdocument
%
\endisadelimdocument
%
\isatagdocument
%
\isamarkupsection{Notation and Robustness%
}
\isamarkuptrue%
%
\endisatagdocument
{\isafolddocument}%
%
\isadelimdocument
%
\endisadelimdocument
\isakeywordONE{definition}\isamarkupfalse%
\ NotTwo\ {\isacharcolon}{\kern0pt}{\isacharcolon}{\kern0pt}\ {\isachardoublequoteopen}E\ {\isasymRightarrow}\ E\ {\isasymRightarrow}\ bool{\isachardoublequoteclose}\isanewline
\ \ \isakeywordTWO{where}\ {\isachardoublequoteopen}NotTwo\ x\ y\ {\isasymlongleftrightarrow}\ Inseparable\ x\ y{\isachardoublequoteclose}\isanewline
\isanewline
\isakeywordONE{lemma}\isamarkupfalse%
\ Phenomenon{\isacharunderscore}{\kern0pt}NotTwo{\isacharunderscore}{\kern0pt}Base{\isacharcolon}{\kern0pt}\ {\isachardoublequoteopen}Phenomenon\ p\ {\isasymLongrightarrow}\ NotTwo\ p\ {\isasymOmega}{\isachardoublequoteclose}\isanewline
%
\isadelimproof
\ \ %
\endisadelimproof
%
\isatagproof
\isakeywordONE{using}\isamarkupfalse%
\ Nonduality\ NotTwo{\isacharunderscore}{\kern0pt}def\ \isakeywordONE{by}\isamarkupfalse%
\ blast%
\endisatagproof
{\isafoldproof}%
%
\isadelimproof
\isanewline
%
\endisadelimproof
\isanewline
\isakeywordONE{lemma}\isamarkupfalse%
\ Any{\isacharunderscore}{\kern0pt}presentation{\isacharunderscore}{\kern0pt}structure{\isacharunderscore}{\kern0pt}preserves{\isacharunderscore}{\kern0pt}NotTwo{\isacharcolon}{\kern0pt}\isanewline
\ \ \isakeywordTWO{assumes}\ {\isachardoublequoteopen}Phenomenon\ x{\isachardoublequoteclose}\ \isakeywordTWO{shows}\ {\isachardoublequoteopen}NotTwo\ x\ {\isasymOmega}{\isachardoublequoteclose}\isanewline
%
\isadelimproof
\ \ %
\endisadelimproof
%
\isatagproof
\isakeywordONE{using}\isamarkupfalse%
\ assms\ Nonduality\ NotTwo{\isacharunderscore}{\kern0pt}def\ \isakeywordONE{by}\isamarkupfalse%
\ blast%
\endisatagproof
{\isafoldproof}%
%
\isadelimproof
\isanewline
%
\endisadelimproof
%
\isadelimtheory
\isanewline
%
\endisadelimtheory
%
\isatagtheory
\isakeywordTWO{end}\isamarkupfalse%
%
\endisatagtheory
{\isafoldtheory}%
%
\isadelimtheory
%
\endisadelimtheory
%
\end{isabellebody}%
\endinput
%:%file=~/Documents/GitHub/unified_Field_theory/Unified_Field_Theory/The_Unique_Ontic_Substrate.thy%:%
%:%10=1%:%
%:%11=1%:%
%:%12=2%:%
%:%13=3%:%
%:%27=43%:%
%:%37=45%:%
%:%38=45%:%
%:%39=46%:%
%:%40=47%:%
%:%41=47%:%
%:%42=48%:%
%:%43=49%:%
%:%44=50%:%
%:%45=51%:%
%:%46=52%:%
%:%47=53%:%
%:%48=53%:%
%:%49=54%:%
%:%50=55%:%
%:%51=56%:%
%:%52=57%:%
%:%53=58%:%
%:%54=59%:%
%:%55=60%:%
%:%56=60%:%
%:%59=61%:%
%:%63=61%:%
%:%64=61%:%
%:%65=61%:%
%:%70=61%:%
%:%73=62%:%
%:%74=63%:%
%:%75=63%:%
%:%76=64%:%
%:%77=65%:%
%:%78=66%:%
%:%79=66%:%
%:%82=67%:%
%:%86=67%:%
%:%87=67%:%
%:%88=67%:%
%:%89=67%:%
%:%94=67%:%
%:%97=68%:%
%:%98=69%:%
%:%99=69%:%
%:%102=70%:%
%:%106=70%:%
%:%107=70%:%
%:%108=70%:%
%:%113=70%:%
%:%116=71%:%
%:%117=72%:%
%:%118=72%:%
%:%120=72%:%
%:%124=72%:%
%:%125=72%:%
%:%139=75%:%
%:%149=77%:%
%:%150=77%:%
%:%151=78%:%
%:%158=79%:%
%:%159=79%:%
%:%160=80%:%
%:%161=80%:%
%:%162=80%:%
%:%163=81%:%
%:%164=81%:%
%:%165=81%:%
%:%166=81%:%
%:%167=82%:%
%:%168=82%:%
%:%169=83%:%
%:%170=83%:%
%:%171=83%:%
%:%172=84%:%
%:%173=84%:%
%:%174=85%:%
%:%175=85%:%
%:%176=85%:%
%:%177=86%:%
%:%192=88%:%
%:%202=90%:%
%:%203=90%:%
%:%204=91%:%
%:%205=92%:%
%:%206=92%:%
%:%207=93%:%
%:%208=94%:%
%:%209=95%:%
%:%210=96%:%
%:%211=97%:%
%:%212=97%:%
%:%213=98%:%
%:%216=99%:%
%:%220=99%:%
%:%221=99%:%
%:%222=99%:%
%:%227=99%:%
%:%230=100%:%
%:%231=101%:%
%:%232=101%:%
%:%233=102%:%
%:%236=103%:%
%:%240=103%:%
%:%241=103%:%
%:%242=103%:%
%:%256=105%:%
%:%266=107%:%
%:%267=107%:%
%:%268=108%:%
%:%269=108%:%
%:%270=109%:%
%:%271=110%:%
%:%272=110%:%
%:%273=111%:%
%:%274=112%:%
%:%275=113%:%
%:%276=114%:%
%:%277=114%:%
%:%278=115%:%
%:%279=116%:%
%:%280=117%:%
%:%281=118%:%
%:%282=119%:%
%:%283=120%:%
%:%284=120%:%
%:%285=121%:%
%:%286=122%:%
%:%293=123%:%
%:%294=123%:%
%:%295=124%:%
%:%296=124%:%
%:%297=124%:%
%:%298=124%:%
%:%299=125%:%
%:%300=125%:%
%:%301=125%:%
%:%302=125%:%
%:%303=126%:%
%:%304=126%:%
%:%305=126%:%
%:%306=126%:%
%:%307=127%:%
%:%322=129%:%
%:%332=131%:%
%:%333=131%:%
%:%334=132%:%
%:%335=133%:%
%:%336=133%:%
%:%337=134%:%
%:%344=136%:%
%:%354=138%:%
%:%355=138%:%
%:%356=139%:%
%:%357=140%:%
%:%358=140%:%
%:%359=141%:%
%:%360=142%:%
%:%361=143%:%
%:%368=145%:%
%:%378=147%:%
%:%379=147%:%
%:%380=148%:%
%:%381=148%:%
%:%382=149%:%
%:%383=149%:%
%:%384=150%:%
%:%385=151%:%
%:%386=151%:%
%:%387=152%:%
%:%388=153%:%
%:%389=154%:%
%:%390=155%:%
%:%397=157%:%
%:%407=159%:%
%:%408=159%:%
%:%409=160%:%
%:%410=160%:%
%:%411=161%:%
%:%412=162%:%
%:%413=162%:%
%:%414=163%:%
%:%415=164%:%
%:%416=165%:%
%:%417=166%:%
%:%418=166%:%
%:%419=167%:%
%:%420=168%:%
%:%423=169%:%
%:%427=169%:%
%:%428=169%:%
%:%429=170%:%
%:%430=170%:%
%:%444=173%:%
%:%454=175%:%
%:%455=175%:%
%:%456=176%:%
%:%457=176%:%
%:%458=177%:%
%:%459=178%:%
%:%460=178%:%
%:%461=179%:%
%:%462=180%:%
%:%463=181%:%
%:%464=182%:%
%:%465=182%:%
%:%466=183%:%
%:%469=184%:%
%:%473=184%:%
%:%474=184%:%
%:%475=184%:%
%:%489=186%:%
%:%499=188%:%
%:%500=188%:%
%:%507=190%:%
%:%517=192%:%
%:%518=192%:%
%:%519=193%:%
%:%520=194%:%
%:%521=195%:%
%:%522=196%:%
%:%523=196%:%
%:%524=197%:%
%:%525=198%:%
%:%526=199%:%
%:%527=199%:%
%:%528=200%:%
%:%531=201%:%
%:%535=201%:%
%:%536=201%:%
%:%537=201%:%
%:%551=203%:%
%:%561=205%:%
%:%562=205%:%
%:%563=206%:%
%:%564=207%:%
%:%565=208%:%
%:%566=209%:%
%:%567=209%:%
%:%568=210%:%
%:%569=211%:%
%:%570=212%:%
%:%571=213%:%
%:%572=214%:%
%:%573=214%:%
%:%574=215%:%
%:%577=216%:%
%:%581=216%:%
%:%582=216%:%
%:%583=216%:%
%:%597=218%:%
%:%607=220%:%
%:%608=220%:%
%:%609=221%:%
%:%610=222%:%
%:%611=222%:%
%:%612=223%:%
%:%613=224%:%
%:%620=226%:%
%:%630=228%:%
%:%631=228%:%
%:%632=229%:%
%:%633=230%:%
%:%634=231%:%
%:%635=231%:%
%:%638=232%:%
%:%642=232%:%
%:%643=232%:%
%:%644=232%:%
%:%649=232%:%
%:%652=233%:%
%:%653=234%:%
%:%654=234%:%
%:%655=235%:%
%:%658=236%:%
%:%662=236%:%
%:%663=236%:%
%:%664=236%:%
%:%669=236%:%
%:%674=237%:%
%:%679=238%:%

%
\begin{isabellebody}%
\setisabellecontext{Unified{\isacharunderscore}{\kern0pt}Field{\isacharunderscore}{\kern0pt}Theory}%
%
\isadelimtheory
%
\endisadelimtheory
%
\isatagtheory
\isakeywordONE{theory}\isamarkupfalse%
\ Unified{\isacharunderscore}{\kern0pt}Field{\isacharunderscore}{\kern0pt}Theory\isanewline
\ \ \isakeywordTWO{imports}\ The{\isacharunderscore}{\kern0pt}Unique{\isacharunderscore}{\kern0pt}Ontic{\isacharunderscore}{\kern0pt}Substrate\isanewline
\isakeywordTWO{begin}%
\endisatagtheory
{\isafoldtheory}%
%
\isadelimtheory
\isanewline
%
\endisadelimtheory
\ \ \isanewline
\ \ \isanewline
\ \ \isanewline
\ \ \isanewline
\ \ \isanewline
\ \ \isanewline
\ \ \isanewline
\ \ \isanewline
\ \ \isanewline
\ \ \isanewline
\isanewline
\ \ \isanewline
\ \ \isakeywordONE{nitpick{\isacharunderscore}{\kern0pt}params}\isamarkupfalse%
\ {\isacharbrackleft}{\kern0pt}user{\isacharunderscore}{\kern0pt}axioms{\isacharcomma}{\kern0pt}\ show{\isacharunderscore}{\kern0pt}all{\isacharcomma}{\kern0pt}\ format\ {\isacharequal}{\kern0pt}\ {\isadigit{3}}{\isacharcomma}{\kern0pt}\ max{\isacharunderscore}{\kern0pt}threads\ {\isacharequal}{\kern0pt}\ {\isadigit{2}}{\isacharcomma}{\kern0pt}\ card\ {\isacharequal}{\kern0pt}\ {\isadigit{1}}{\isacharcomma}{\kern0pt}{\isadigit{2}}{\isacharcomma}{\kern0pt}{\isadigit{3}}{\isacharcomma}{\kern0pt}{\isadigit{4}}{\isacharcomma}{\kern0pt}{\isadigit{5}}{\isacharbrackright}{\kern0pt}%
\isadelimdocument
%
\endisadelimdocument
%
\isatagdocument
%
\isamarkupsection{Quantum Fields as Presentation Channels%
}
\isamarkuptrue%
%
\endisatagdocument
{\isafolddocument}%
%
\isadelimdocument
%
\endisadelimdocument
\ \ \isakeywordONE{typedecl}\isamarkupfalse%
\ FieldType\ \ \ \isanewline
\isanewline
\ \ \isakeywordONE{consts}\isamarkupfalse%
\isanewline
\ \ \ \ FieldChannel\ {\isacharcolon}{\kern0pt}{\isacharcolon}{\kern0pt}\ {\isachardoublequoteopen}FieldType\ {\isasymRightarrow}\ bool{\isachardoublequoteclose}\isanewline
\ \ \ \ Excitation\ \ \ {\isacharcolon}{\kern0pt}{\isacharcolon}{\kern0pt}\ {\isachardoublequoteopen}E\ {\isasymRightarrow}\ FieldType\ {\isasymRightarrow}\ bool{\isachardoublequoteclose}\ \ \isanewline
\ \ \ \ GroundState\ \ {\isacharcolon}{\kern0pt}{\isacharcolon}{\kern0pt}\ {\isachardoublequoteopen}FieldType\ {\isasymRightarrow}\ bool{\isachardoublequoteclose}\ \ \ \ \ \ \ \ \ \ \ \ \isanewline
\isanewline
\ \ \isakeywordONE{axiomatization}\isamarkupfalse%
\ \isakeywordTWO{where}\isanewline
\ \ \ \ FC{\isadigit{2}}{\isacharunderscore}{\kern0pt}excitations{\isacharunderscore}{\kern0pt}are{\isacharunderscore}{\kern0pt}phenomena{\isacharcolon}{\kern0pt}\isanewline
\ \ \ \ \ \ {\isachardoublequoteopen}{\isasymforall}e\ ft{\isachardot}{\kern0pt}\ Excitation\ e\ ft\ {\isasymlongrightarrow}\ Phenomenon\ e\ {\isasymand}\ FieldChannel\ ft{\isachardoublequoteclose}\ \isakeywordTWO{and}\isanewline
\ \ \ \ FC{\isadigit{3}}{\isacharunderscore}{\kern0pt}ground{\isacharunderscore}{\kern0pt}state{\isacharunderscore}{\kern0pt}exists{\isacharcolon}{\kern0pt}\isanewline
\ \ \ \ \ \ {\isachardoublequoteopen}{\isasymforall}ft{\isachardot}{\kern0pt}\ FieldChannel\ ft\ {\isasymlongrightarrow}\ GroundState\ ft{\isachardoublequoteclose}\isanewline
\isanewline
\ \ \isakeywordONE{lemma}\isamarkupfalse%
\ Field{\isacharunderscore}{\kern0pt}channels{\isacharunderscore}{\kern0pt}structure{\isacharunderscore}{\kern0pt}presentation{\isacharcolon}{\kern0pt}\isanewline
\ \ \ \ \isakeywordTWO{assumes}\ {\isachardoublequoteopen}Excitation\ e\ ft{\isachardoublequoteclose}\isanewline
\ \ \ \ \isakeywordTWO{shows}\ {\isachardoublequoteopen}Inseparable\ e\ {\isasymOmega}{\isachardoublequoteclose}\isanewline
%
\isadelimproof
\ \ \ \ %
\endisadelimproof
%
\isatagproof
\isakeywordONE{using}\isamarkupfalse%
\ assms\ FC{\isadigit{2}}{\isacharunderscore}{\kern0pt}excitations{\isacharunderscore}{\kern0pt}are{\isacharunderscore}{\kern0pt}phenomena\ Nonduality\ \isakeywordONE{by}\isamarkupfalse%
\ blast%
\endisatagproof
{\isafoldproof}%
%
\isadelimproof
\isanewline
%
\endisadelimproof
\isanewline
\ \ \isakeywordONE{theorem}\isamarkupfalse%
\ Field{\isacharunderscore}{\kern0pt}excitations{\isacharunderscore}{\kern0pt}unified{\isacharcolon}{\kern0pt}\isanewline
\ \ \ \ {\isachardoublequoteopen}{\isasymforall}e{\isadigit{1}}\ e{\isadigit{2}}\ ft{\isachardot}{\kern0pt}\ Excitation\ e{\isadigit{1}}\ ft\ {\isasymand}\ Excitation\ e{\isadigit{2}}\ ft\ {\isasymlongrightarrow}\isanewline
\ \ \ \ \ \ \ Inseparable\ e{\isadigit{1}}\ {\isasymOmega}\ {\isasymand}\ Inseparable\ e{\isadigit{2}}\ {\isasymOmega}{\isachardoublequoteclose}\isanewline
%
\isadelimproof
\ \ \ \ %
\endisadelimproof
%
\isatagproof
\isakeywordONE{using}\isamarkupfalse%
\ Field{\isacharunderscore}{\kern0pt}channels{\isacharunderscore}{\kern0pt}structure{\isacharunderscore}{\kern0pt}presentation\ \isakeywordONE{by}\isamarkupfalse%
\ blast%
\endisatagproof
{\isafoldproof}%
%
\isadelimproof
%
\endisadelimproof
%
\isadelimdocument
%
\endisadelimdocument
%
\isatagdocument
%
\isamarkupsection{Gauge Structure as Presentation Indexing%
}
\isamarkuptrue%
%
\endisatagdocument
{\isafolddocument}%
%
\isadelimdocument
%
\endisadelimdocument
\ \ \isakeywordONE{typedecl}\isamarkupfalse%
\ GaugeGroup\ \ \isanewline
\isanewline
\ \ \isakeywordONE{consts}\isamarkupfalse%
\isanewline
\ \ \ \ GaugeDomain\ {\isacharcolon}{\kern0pt}{\isacharcolon}{\kern0pt}\ {\isachardoublequoteopen}GaugeGroup\ {\isasymRightarrow}\ FieldType\ set{\isachardoublequoteclose}\ \ \ \isanewline
\ \ \ \ Unified\ \ \ \ \ {\isacharcolon}{\kern0pt}{\isacharcolon}{\kern0pt}\ {\isachardoublequoteopen}GaugeGroup\ {\isasymRightarrow}\ GaugeGroup\ set\ {\isasymRightarrow}\ bool{\isachardoublequoteclose}\ \ \isanewline
\ \ \ \ IndexScheme\ {\isacharcolon}{\kern0pt}{\isacharcolon}{\kern0pt}\ {\isachardoublequoteopen}GaugeGroup\ {\isasymRightarrow}\ E\ {\isasymRightarrow}\ bool{\isachardoublequoteclose}\ \ \ \ \ \ \isanewline
\isanewline
\ \ \isakeywordONE{axiomatization}\isamarkupfalse%
\ \isakeywordTWO{where}\isanewline
\ \ \ \ G{\isadigit{1}}{\isacharunderscore}{\kern0pt}gauge{\isacharunderscore}{\kern0pt}indexes{\isacharunderscore}{\kern0pt}phenomena{\isacharcolon}{\kern0pt}\isanewline
\ \ \ \ \ \ {\isachardoublequoteopen}{\isasymforall}gg\ e{\isachardot}{\kern0pt}\ IndexScheme\ gg\ e\ {\isasymlongrightarrow}\ Phenomenon\ e{\isachardoublequoteclose}\ \isakeywordTWO{and}\isanewline
\ \ \ \ G{\isadigit{2}}{\isacharunderscore}{\kern0pt}unified{\isacharunderscore}{\kern0pt}preserves{\isacharunderscore}{\kern0pt}indexing{\isacharcolon}{\kern0pt}\isanewline
\ \ \ \ \ \ {\isachardoublequoteopen}{\isasymforall}gg\ subgroups\ e{\isachardot}{\kern0pt}\ Unified\ gg\ subgroups\ {\isasymand}\ {\isacharparenleft}{\kern0pt}{\isasymexists}sg\ {\isasymin}\ subgroups{\isachardot}{\kern0pt}\ IndexScheme\ sg\ e{\isacharparenright}{\kern0pt}\isanewline
\ \ \ \ \ \ \ \ \ {\isasymlongrightarrow}\ IndexScheme\ gg\ e{\isachardoublequoteclose}\ \isakeywordTWO{and}\isanewline
\ \ \ \ G{\isadigit{3}}{\isacharunderscore}{\kern0pt}gauge{\isacharunderscore}{\kern0pt}domain{\isacharunderscore}{\kern0pt}correspondence{\isacharcolon}{\kern0pt}\isanewline
\ \ \ \ \ \ {\isachardoublequoteopen}{\isasymforall}gg\ ft\ e{\isachardot}{\kern0pt}\ IndexScheme\ gg\ e\ {\isasymand}\ Excitation\ e\ ft\ {\isasymlongrightarrow}\ ft\ {\isasymin}\ GaugeDomain\ gg{\isachardoublequoteclose}\isanewline
\isanewline
\ \ \isakeywordONE{lemma}\isamarkupfalse%
\ Gauge{\isacharunderscore}{\kern0pt}indexing{\isacharunderscore}{\kern0pt}preserves{\isacharunderscore}{\kern0pt}nonduality{\isacharcolon}{\kern0pt}\isanewline
\ \ \ \ \isakeywordTWO{assumes}\ {\isachardoublequoteopen}IndexScheme\ gg\ e{\isachardoublequoteclose}\isanewline
\ \ \ \ \isakeywordTWO{shows}\ {\isachardoublequoteopen}Inseparable\ e\ {\isasymOmega}{\isachardoublequoteclose}\isanewline
%
\isadelimproof
\ \ \ \ %
\endisadelimproof
%
\isatagproof
\isakeywordONE{using}\isamarkupfalse%
\ assms\ G{\isadigit{1}}{\isacharunderscore}{\kern0pt}gauge{\isacharunderscore}{\kern0pt}indexes{\isacharunderscore}{\kern0pt}phenomena\ Nonduality\ \isakeywordONE{by}\isamarkupfalse%
\ blast%
\endisatagproof
{\isafoldproof}%
%
\isadelimproof
\isanewline
%
\endisadelimproof
\isanewline
\ \ \isakeywordONE{theorem}\isamarkupfalse%
\ Gauge{\isacharunderscore}{\kern0pt}unification{\isacharunderscore}{\kern0pt}ontological{\isacharcolon}{\kern0pt}\isanewline
\ \ \ \ {\isachardoublequoteopen}{\isasymforall}gg\ subgroups\ e{\isachardot}{\kern0pt}\ Unified\ gg\ subgroups\ {\isasymand}\ {\isacharparenleft}{\kern0pt}{\isasymexists}sg\ {\isasymin}\ subgroups{\isachardot}{\kern0pt}\ IndexScheme\ sg\ e{\isacharparenright}{\kern0pt}\isanewline
\ \ \ \ \ \ \ {\isasymlongrightarrow}\ Inseparable\ e\ {\isasymOmega}{\isachardoublequoteclose}\isanewline
%
\isadelimproof
\ \ \ \ %
\endisadelimproof
%
\isatagproof
\isakeywordONE{using}\isamarkupfalse%
\ G{\isadigit{2}}{\isacharunderscore}{\kern0pt}unified{\isacharunderscore}{\kern0pt}preserves{\isacharunderscore}{\kern0pt}indexing\ Gauge{\isacharunderscore}{\kern0pt}indexing{\isacharunderscore}{\kern0pt}preserves{\isacharunderscore}{\kern0pt}nonduality\ \isakeywordONE{by}\isamarkupfalse%
\ blast%
\endisatagproof
{\isafoldproof}%
%
\isadelimproof
%
\endisadelimproof
%
\isadelimdocument
%
\endisadelimdocument
%
\isatagdocument
%
\isamarkupsection{Force Phenomena as Presentation Modes%
}
\isamarkuptrue%
%
\endisatagdocument
{\isafolddocument}%
%
\isadelimdocument
%
\endisadelimdocument
\ \ \isakeywordONE{datatype}\isamarkupfalse%
\ ForceType\ {\isacharequal}{\kern0pt}\ Electromagnetic\ {\isacharbar}{\kern0pt}\ Weak\ {\isacharbar}{\kern0pt}\ Strong\ {\isacharbar}{\kern0pt}\ Gravitational\isanewline
\isanewline
\ \ \isakeywordONE{consts}\isamarkupfalse%
\isanewline
\ \ \ \ ForcePresentation\ {\isacharcolon}{\kern0pt}{\isacharcolon}{\kern0pt}\ {\isachardoublequoteopen}E\ {\isasymRightarrow}\ ForceType\ {\isasymRightarrow}\ bool{\isachardoublequoteclose}\ \ \isanewline
\ \ \ \ UnifiedForce\ \ \ \ \ \ {\isacharcolon}{\kern0pt}{\isacharcolon}{\kern0pt}\ {\isachardoublequoteopen}E\ {\isasymRightarrow}\ bool{\isachardoublequoteclose}\ \ \ \ \ \ \ \ \ \ \ \ \ \ \ \ \ \ \ \ \isanewline
\isanewline
\ \ \isakeywordONE{axiomatization}\isamarkupfalse%
\ \isakeywordTWO{where}\isanewline
\ \ \ \ F{\isadigit{1}}{\isacharunderscore}{\kern0pt}forces{\isacharunderscore}{\kern0pt}phenomenal{\isacharcolon}{\kern0pt}\isanewline
\ \ \ \ \ \ {\isachardoublequoteopen}{\isasymforall}e\ ft{\isachardot}{\kern0pt}\ ForcePresentation\ e\ ft\ {\isasymlongrightarrow}\ Phenomenon\ e{\isachardoublequoteclose}\ \isakeywordTWO{and}\isanewline
\ \ \ \ F{\isadigit{2}}{\isacharunderscore}{\kern0pt}unified{\isacharunderscore}{\kern0pt}includes{\isacharunderscore}{\kern0pt}all{\isacharcolon}{\kern0pt}\isanewline
\ \ \ \ \ \ {\isachardoublequoteopen}{\isasymforall}e{\isachardot}{\kern0pt}\ UnifiedForce\ e\ {\isasymlongrightarrow}\isanewline
\ \ \ \ \ \ \ \ \ {\isacharparenleft}{\kern0pt}ForcePresentation\ e\ Electromagnetic\ {\isasymor}\ ForcePresentation\ e\ Weak\ {\isasymor}\isanewline
\ \ \ \ \ \ \ \ \ \ ForcePresentation\ e\ Strong\ {\isasymor}\ ForcePresentation\ e\ Gravitational{\isacharparenright}{\kern0pt}{\isachardoublequoteclose}\ \isakeywordTWO{and}\isanewline
\ \ \ \ F{\isadigit{3}}{\isacharunderscore}{\kern0pt}forces{\isacharunderscore}{\kern0pt}via{\isacharunderscore}{\kern0pt}presentation{\isacharcolon}{\kern0pt}\isanewline
\ \ \ \ \ \ {\isachardoublequoteopen}{\isasymforall}e\ ft{\isachardot}{\kern0pt}\ ForcePresentation\ e\ ft\ {\isasymlongrightarrow}\ Presents\ e\ {\isasymOmega}{\isachardoublequoteclose}\isanewline
\isanewline
\ \ \isakeywordONE{lemma}\isamarkupfalse%
\ Force{\isacharunderscore}{\kern0pt}phenomena{\isacharunderscore}{\kern0pt}nondual{\isacharcolon}{\kern0pt}\isanewline
\ \ \ \ \isakeywordTWO{assumes}\ {\isachardoublequoteopen}ForcePresentation\ e\ ft{\isachardoublequoteclose}\isanewline
\ \ \ \ \isakeywordTWO{shows}\ {\isachardoublequoteopen}Inseparable\ e\ {\isasymOmega}{\isachardoublequoteclose}\isanewline
%
\isadelimproof
\ \ \ \ %
\endisadelimproof
%
\isatagproof
\isakeywordONE{using}\isamarkupfalse%
\ assms\ F{\isadigit{1}}{\isacharunderscore}{\kern0pt}forces{\isacharunderscore}{\kern0pt}phenomenal\ Nonduality\ \isakeywordONE{by}\isamarkupfalse%
\ blast%
\endisatagproof
{\isafoldproof}%
%
\isadelimproof
\isanewline
%
\endisadelimproof
\isanewline
\ \ \isakeywordONE{theorem}\isamarkupfalse%
\ Force{\isacharunderscore}{\kern0pt}unification{\isacharunderscore}{\kern0pt}via{\isacharunderscore}{\kern0pt}substrate{\isacharcolon}{\kern0pt}\isanewline
\ \ \ \ {\isachardoublequoteopen}{\isasymforall}e{\isadigit{1}}\ e{\isadigit{2}}\ ft{\isadigit{1}}\ ft{\isadigit{2}}{\isachardot}{\kern0pt}\ ForcePresentation\ e{\isadigit{1}}\ ft{\isadigit{1}}\ {\isasymand}\ ForcePresentation\ e{\isadigit{2}}\ ft{\isadigit{2}}\isanewline
\ \ \ \ \ \ \ {\isasymlongrightarrow}\ {\isacharparenleft}{\kern0pt}{\isasymexists}s{\isachardot}{\kern0pt}\ Substrate\ s\ {\isasymand}\ Presents\ e{\isadigit{1}}\ s\ {\isasymand}\ Presents\ e{\isadigit{2}}\ s{\isacharparenright}{\kern0pt}{\isachardoublequoteclose}\isanewline
%
\isadelimproof
\ \ %
\endisadelimproof
%
\isatagproof
\isakeywordONE{proof}\isamarkupfalse%
\ {\isacharparenleft}{\kern0pt}intro\ allI\ impI{\isacharparenright}{\kern0pt}\isanewline
\ \ \ \ \isakeywordTHREE{fix}\isamarkupfalse%
\ e{\isadigit{1}}\ e{\isadigit{2}}\ ft{\isadigit{1}}\ ft{\isadigit{2}}\isanewline
\ \ \ \ \isakeywordTHREE{assume}\isamarkupfalse%
\ {\isachardoublequoteopen}ForcePresentation\ e{\isadigit{1}}\ ft{\isadigit{1}}\ {\isasymand}\ ForcePresentation\ e{\isadigit{2}}\ ft{\isadigit{2}}{\isachardoublequoteclose}\isanewline
\ \ \ \ \isakeywordONE{hence}\isamarkupfalse%
\ {\isachardoublequoteopen}Phenomenon\ e{\isadigit{1}}\ {\isasymand}\ Phenomenon\ e{\isadigit{2}}{\isachardoublequoteclose}\ \isakeywordONE{using}\isamarkupfalse%
\ F{\isadigit{1}}{\isacharunderscore}{\kern0pt}forces{\isacharunderscore}{\kern0pt}phenomenal\ \isakeywordONE{by}\isamarkupfalse%
\ blast\isanewline
\ \ \ \ \isakeywordTHREE{thus}\isamarkupfalse%
\ {\isachardoublequoteopen}{\isacharparenleft}{\kern0pt}{\isasymexists}s{\isachardot}{\kern0pt}\ Substrate\ s\ {\isasymand}\ Presents\ e{\isadigit{1}}\ s\ {\isasymand}\ Presents\ e{\isadigit{2}}\ s{\isacharparenright}{\kern0pt}{\isachardoublequoteclose}\isanewline
\ \ \ \ \ \ \isakeywordONE{using}\isamarkupfalse%
\ substrate{\isacharunderscore}{\kern0pt}Omega\ A{\isadigit{4}}{\isacharunderscore}{\kern0pt}presentation\ \isakeywordONE{by}\isamarkupfalse%
\ blast\isanewline
\ \ \isakeywordONE{qed}\isamarkupfalse%
%
\endisatagproof
{\isafoldproof}%
%
\isadelimproof
%
\endisadelimproof
%
\isadelimdocument
%
\endisadelimdocument
%
\isatagdocument
%
\isamarkupsection{Entanglement Structure from Substrate Unity%
}
\isamarkuptrue%
%
\endisatagdocument
{\isafolddocument}%
%
\isadelimdocument
%
\endisadelimdocument
\ \ \isakeywordONE{consts}\isamarkupfalse%
\isanewline
\ \ \ \ Entangled\ {\isacharcolon}{\kern0pt}{\isacharcolon}{\kern0pt}\ {\isachardoublequoteopen}E\ {\isasymRightarrow}\ E\ {\isasymRightarrow}\ bool{\isachardoublequoteclose}\ \ \isanewline
\ \ \ \ EntCorr\ \ \ {\isacharcolon}{\kern0pt}{\isacharcolon}{\kern0pt}\ {\isachardoublequoteopen}E\ {\isasymRightarrow}\ E\ {\isasymRightarrow}\ Q{\isachardoublequoteclose}\ \ \ \ \isanewline
\isanewline
\ \ \isakeywordONE{axiomatization}\isamarkupfalse%
\ \isakeywordTWO{where}\isanewline
\ \ \ \ ENT{\isadigit{1}}{\isacharunderscore}{\kern0pt}entangled{\isacharunderscore}{\kern0pt}phenomena{\isacharcolon}{\kern0pt}\isanewline
\ \ \ \ \ \ {\isachardoublequoteopen}{\isasymforall}e{\isadigit{1}}\ e{\isadigit{2}}{\isachardot}{\kern0pt}\ Entangled\ e{\isadigit{1}}\ e{\isadigit{2}}\ {\isasymlongrightarrow}\ Phenomenon\ e{\isadigit{1}}\ {\isasymand}\ Phenomenon\ e{\isadigit{2}}{\isachardoublequoteclose}\ \isakeywordTWO{and}\isanewline
\ \ \ \ ENT{\isadigit{2}}{\isacharunderscore}{\kern0pt}entanglement{\isacharunderscore}{\kern0pt}symmetric{\isacharcolon}{\kern0pt}\isanewline
\ \ \ \ \ \ {\isachardoublequoteopen}{\isasymforall}e{\isadigit{1}}\ e{\isadigit{2}}{\isachardot}{\kern0pt}\ Entangled\ e{\isadigit{1}}\ e{\isadigit{2}}\ {\isasymlongleftrightarrow}\ Entangled\ e{\isadigit{2}}\ e{\isadigit{1}}{\isachardoublequoteclose}\ \isakeywordTWO{and}\isanewline
\ \ \ \ ENT{\isadigit{3}}{\isacharunderscore}{\kern0pt}substrate{\isacharunderscore}{\kern0pt}unity{\isacharcolon}{\kern0pt}\isanewline
\ \ \ \ \ \ {\isachardoublequoteopen}{\isasymforall}e{\isadigit{1}}\ e{\isadigit{2}}{\isachardot}{\kern0pt}\ Entangled\ e{\isadigit{1}}\ e{\isadigit{2}}\ {\isasymlongrightarrow}\isanewline
\ \ \ \ \ \ \ \ \ {\isacharparenleft}{\kern0pt}{\isasymexists}s{\isachardot}{\kern0pt}\ Substrate\ s\ {\isasymand}\ Presents\ e{\isadigit{1}}\ s\ {\isasymand}\ Presents\ e{\isadigit{2}}\ s\ {\isasymand}\ s\ {\isacharequal}{\kern0pt}\ {\isasymOmega}{\isacharparenright}{\kern0pt}{\isachardoublequoteclose}\ \isakeywordTWO{and}\isanewline
\ \ \ \ ENT{\isadigit{4}}{\isacharunderscore}{\kern0pt}correlation{\isacharunderscore}{\kern0pt}nonneg{\isacharcolon}{\kern0pt}\isanewline
\ \ \ \ \ \ {\isachardoublequoteopen}{\isasymforall}e{\isadigit{1}}\ e{\isadigit{2}}{\isachardot}{\kern0pt}\ Entangled\ e{\isadigit{1}}\ e{\isadigit{2}}\ {\isasymlongrightarrow}\ Nonneg\ {\isacharparenleft}{\kern0pt}EntCorr\ e{\isadigit{1}}\ e{\isadigit{2}}{\isacharparenright}{\kern0pt}{\isachardoublequoteclose}\isanewline
\isanewline
\ \ \isakeywordONE{theorem}\isamarkupfalse%
\ Entanglement{\isacharunderscore}{\kern0pt}via{\isacharunderscore}{\kern0pt}nonduality{\isacharcolon}{\kern0pt}\isanewline
\ \ \ \ {\isachardoublequoteopen}{\isasymforall}e{\isadigit{1}}\ e{\isadigit{2}}{\isachardot}{\kern0pt}\ Entangled\ e{\isadigit{1}}\ e{\isadigit{2}}\ {\isasymlongrightarrow}\ Inseparable\ e{\isadigit{1}}\ {\isasymOmega}\ {\isasymand}\ Inseparable\ e{\isadigit{2}}\ {\isasymOmega}{\isachardoublequoteclose}\isanewline
%
\isadelimproof
\ \ %
\endisadelimproof
%
\isatagproof
\isakeywordONE{proof}\isamarkupfalse%
\ {\isacharparenleft}{\kern0pt}intro\ allI\ impI{\isacharparenright}{\kern0pt}\isanewline
\ \ \ \ \isakeywordTHREE{fix}\isamarkupfalse%
\ e{\isadigit{1}}\ e{\isadigit{2}}\isanewline
\ \ \ \ \isakeywordTHREE{assume}\isamarkupfalse%
\ H{\isacharcolon}{\kern0pt}\ {\isachardoublequoteopen}Entangled\ e{\isadigit{1}}\ e{\isadigit{2}}{\isachardoublequoteclose}\isanewline
\ \ \ \ \isakeywordONE{hence}\isamarkupfalse%
\ {\isachardoublequoteopen}Phenomenon\ e{\isadigit{1}}\ {\isasymand}\ Phenomenon\ e{\isadigit{2}}{\isachardoublequoteclose}\ \isakeywordONE{using}\isamarkupfalse%
\ ENT{\isadigit{1}}{\isacharunderscore}{\kern0pt}entangled{\isacharunderscore}{\kern0pt}phenomena\ \isakeywordONE{by}\isamarkupfalse%
\ blast\isanewline
\ \ \ \ \isakeywordTHREE{thus}\isamarkupfalse%
\ {\isachardoublequoteopen}Inseparable\ e{\isadigit{1}}\ {\isasymOmega}\ {\isasymand}\ Inseparable\ e{\isadigit{2}}\ {\isasymOmega}{\isachardoublequoteclose}\ \isakeywordONE{using}\isamarkupfalse%
\ Nonduality\ \isakeywordONE{by}\isamarkupfalse%
\ blast\isanewline
\ \ \isakeywordONE{qed}\isamarkupfalse%
%
\endisatagproof
{\isafoldproof}%
%
\isadelimproof
\isanewline
%
\endisadelimproof
\isanewline
\ \ \isakeywordONE{lemma}\isamarkupfalse%
\ Entanglement{\isacharunderscore}{\kern0pt}nonlocal{\isacharunderscore}{\kern0pt}via{\isacharunderscore}{\kern0pt}substrate{\isacharcolon}{\kern0pt}\isanewline
\ \ \ \ \isakeywordTWO{assumes}\ {\isachardoublequoteopen}Entangled\ e{\isadigit{1}}\ e{\isadigit{2}}{\isachardoublequoteclose}\isanewline
\ \ \ \ \ \ \ \ \isakeywordTWO{and}\ {\isachardoublequoteopen}coord\ f\ e{\isadigit{1}}\ {\isacharequal}{\kern0pt}\ Some\ r{\isadigit{1}}{\isachardoublequoteclose}\isanewline
\ \ \ \ \ \ \ \ \isakeywordTWO{and}\ {\isachardoublequoteopen}coord\ f\ e{\isadigit{2}}\ {\isacharequal}{\kern0pt}\ Some\ r{\isadigit{2}}{\isachardoublequoteclose}\isanewline
\ \ \ \ \isakeywordTWO{shows}\ {\isachardoublequoteopen}Inseparable\ e{\isadigit{1}}\ {\isasymOmega}\ {\isasymand}\ Inseparable\ e{\isadigit{2}}\ {\isasymOmega}{\isachardoublequoteclose}\isanewline
%
\isadelimproof
\ \ \ \ %
\endisadelimproof
%
\isatagproof
\isakeywordONE{using}\isamarkupfalse%
\ assms\ ENT{\isadigit{1}}{\isacharunderscore}{\kern0pt}entangled{\isacharunderscore}{\kern0pt}phenomena\ Nonduality\ \isakeywordONE{by}\isamarkupfalse%
\ blast%
\endisatagproof
{\isafoldproof}%
%
\isadelimproof
%
\endisadelimproof
%
\isadelimdocument
%
\endisadelimdocument
%
\isatagdocument
%
\isamarkupsection{Information-Theoretic Foundations%
}
\isamarkuptrue%
%
\endisatagdocument
{\isafolddocument}%
%
\isadelimdocument
%
\endisadelimdocument
\ \ \isakeywordONE{typedecl}\isamarkupfalse%
\ Label\ \ \isanewline
\isanewline
\ \ \isakeywordONE{consts}\isamarkupfalse%
\isanewline
\ \ \ \ InfoGeometry\ {\isacharcolon}{\kern0pt}{\isacharcolon}{\kern0pt}\ {\isachardoublequoteopen}E\ set\ {\isasymRightarrow}\ Q{\isachardoublequoteclose}\ \ \isanewline
\ \ \ \ FundConst\ \ \ \ {\isacharcolon}{\kern0pt}{\isacharcolon}{\kern0pt}\ {\isachardoublequoteopen}Label\ {\isasymRightarrow}\ Q{\isachardoublequoteclose}\ \ \isanewline
\ \ \ \ Info\ \ \ \ \ \ \ \ \ {\isacharcolon}{\kern0pt}{\isacharcolon}{\kern0pt}\ {\isachardoublequoteopen}E\ {\isasymRightarrow}\ Q{\isachardoublequoteclose}\ \ \ \ \ \ \isanewline
\ \ \ \ ConstAlpha\ \ \ {\isacharcolon}{\kern0pt}{\isacharcolon}{\kern0pt}\ {\isachardoublequoteopen}Label{\isachardoublequoteclose}\ \ \ \ \ \ \ \ \ \ \ \ \ \ \isanewline
\ \ \ \ ConstPlanck\ \ {\isacharcolon}{\kern0pt}{\isacharcolon}{\kern0pt}\ {\isachardoublequoteopen}Label{\isachardoublequoteclose}\ \ \ \ \ \ \ \ \ \ \ \ \ \ \isanewline
\isanewline
\ \ \isakeywordONE{axiomatization}\isamarkupfalse%
\ \isakeywordTWO{where}\isanewline
\ \ \ \ IG{\isadigit{1}}{\isacharunderscore}{\kern0pt}info{\isacharunderscore}{\kern0pt}over{\isacharunderscore}{\kern0pt}phenomena{\isacharcolon}{\kern0pt}\isanewline
\ \ \ \ \ \ {\isachardoublequoteopen}{\isasymforall}es{\isachardot}{\kern0pt}\ {\isacharparenleft}{\kern0pt}{\isasymforall}e\ {\isasymin}\ es{\isachardot}{\kern0pt}\ Phenomenon\ e{\isacharparenright}{\kern0pt}\ {\isasymlongrightarrow}\ Nonneg\ {\isacharparenleft}{\kern0pt}InfoGeometry\ es{\isacharparenright}{\kern0pt}{\isachardoublequoteclose}\ \isakeywordTWO{and}\isanewline
\ \ \ \ IG{\isadigit{2}}{\isacharunderscore}{\kern0pt}constants{\isacharunderscore}{\kern0pt}nonneg{\isacharcolon}{\kern0pt}\isanewline
\ \ \ \ \ \ {\isachardoublequoteopen}{\isasymforall}name{\isachardot}{\kern0pt}\ Nonneg\ {\isacharparenleft}{\kern0pt}FundConst\ name{\isacharparenright}{\kern0pt}{\isachardoublequoteclose}\ \isakeywordTWO{and}\isanewline
\ \ \ \ IG{\isadigit{3}}{\isacharunderscore}{\kern0pt}holographic{\isacharunderscore}{\kern0pt}bound{\isacharcolon}{\kern0pt}\isanewline
\ \ \ \ \ \ {\isachardoublequoteopen}{\isasymforall}es\ e{\isachardot}{\kern0pt}\ e\ {\isasymin}\ es\ {\isasymand}\ Phenomenon\ e\ {\isasymlongrightarrow}\ {\isacharparenleft}{\kern0pt}{\isasymexists}q{\isachardot}{\kern0pt}\ LT\ {\isacharparenleft}{\kern0pt}Info\ e{\isacharparenright}{\kern0pt}\ q\ {\isasymand}\ LT\ q\ {\isacharparenleft}{\kern0pt}InfoGeometry\ es{\isacharparenright}{\kern0pt}{\isacharparenright}{\kern0pt}{\isachardoublequoteclose}\isanewline
\isanewline
\ \ \isakeywordONE{lemma}\isamarkupfalse%
\ Information{\isacharunderscore}{\kern0pt}nonreifying{\isacharunderscore}{\kern0pt}collective{\isacharcolon}{\kern0pt}\isanewline
\ \ \ \ \isakeywordTWO{assumes}\ {\isachardoublequoteopen}{\isasymforall}e\ {\isasymin}\ es{\isachardot}{\kern0pt}\ Phenomenon\ e{\isachardoublequoteclose}\isanewline
\ \ \ \ \isakeywordTWO{shows}\ {\isachardoublequoteopen}{\isasymforall}e\ {\isasymin}\ es{\isachardot}{\kern0pt}\ Inseparable\ e\ {\isasymOmega}{\isachardoublequoteclose}\isanewline
%
\isadelimproof
\ \ \ \ %
\endisadelimproof
%
\isatagproof
\isakeywordONE{using}\isamarkupfalse%
\ assms\ Nonduality\ \isakeywordONE{by}\isamarkupfalse%
\ blast%
\endisatagproof
{\isafoldproof}%
%
\isadelimproof
\isanewline
%
\endisadelimproof
\isanewline
\ \ \isakeywordONE{theorem}\isamarkupfalse%
\ Constants{\isacharunderscore}{\kern0pt}encode{\isacharunderscore}{\kern0pt}presentation{\isacharunderscore}{\kern0pt}structure{\isacharcolon}{\kern0pt}\isanewline
\ \ \ \ {\isachardoublequoteopen}{\isasymforall}name{\isachardot}{\kern0pt}\ Nonneg\ {\isacharparenleft}{\kern0pt}FundConst\ name{\isacharparenright}{\kern0pt}{\isachardoublequoteclose}\isanewline
%
\isadelimproof
\ \ \ \ %
\endisadelimproof
%
\isatagproof
\isakeywordONE{using}\isamarkupfalse%
\ IG{\isadigit{2}}{\isacharunderscore}{\kern0pt}constants{\isacharunderscore}{\kern0pt}nonneg\ \isakeywordONE{by}\isamarkupfalse%
\ blast%
\endisatagproof
{\isafoldproof}%
%
\isadelimproof
%
\endisadelimproof
%
\isadelimdocument
%
\endisadelimdocument
%
\isatagdocument
%
\isamarkupsection{Spacetime Geometry from Presentation Structure%
}
\isamarkuptrue%
%
\endisatagdocument
{\isafolddocument}%
%
\isadelimdocument
%
\endisadelimdocument
\ \ \isakeywordONE{consts}\isamarkupfalse%
\isanewline
\ \ \ \ Curvature\ {\isacharcolon}{\kern0pt}{\isacharcolon}{\kern0pt}\ {\isachardoublequoteopen}Frame\ {\isasymRightarrow}\ E\ {\isasymRightarrow}\ Q{\isachardoublequoteclose}\ \ \isanewline
\ \ \ \ GravField\ {\isacharcolon}{\kern0pt}{\isacharcolon}{\kern0pt}\ {\isachardoublequoteopen}E\ {\isasymRightarrow}\ bool{\isachardoublequoteclose}\ \ \ \ \ \ \ \ \ \ \isanewline
\isanewline
\ \ \isakeywordONE{axiomatization}\isamarkupfalse%
\ \isakeywordTWO{where}\isanewline
\ \ \ \ ST{\isadigit{1}}{\isacharunderscore}{\kern0pt}curvature{\isacharunderscore}{\kern0pt}for{\isacharunderscore}{\kern0pt}phenomena{\isacharcolon}{\kern0pt}\isanewline
\ \ \ \ \ \ {\isachardoublequoteopen}{\isasymforall}f\ e\ q{\isachardot}{\kern0pt}\ coord\ f\ e\ {\isacharequal}{\kern0pt}\ Some\ q\ {\isasymlongrightarrow}\ Nonneg\ {\isacharparenleft}{\kern0pt}Curvature\ f\ e{\isacharparenright}{\kern0pt}{\isachardoublequoteclose}\ \isakeywordTWO{and}\isanewline
\ \ \ \ ST{\isadigit{2}}{\isacharunderscore}{\kern0pt}gravity{\isacharunderscore}{\kern0pt}relational{\isacharcolon}{\kern0pt}\isanewline
\ \ \ \ \ \ {\isachardoublequoteopen}{\isasymforall}e{\isachardot}{\kern0pt}\ GravField\ e\ {\isasymlongrightarrow}\ Phenomenon\ e\ {\isasymand}\isanewline
\ \ \ \ \ \ \ \ \ {\isacharparenleft}{\kern0pt}{\isasymexists}e{\isadigit{1}}\ e{\isadigit{2}}{\isachardot}{\kern0pt}\ Phenomenon\ e{\isadigit{1}}\ {\isasymand}\ Phenomenon\ e{\isadigit{2}}\ {\isasymand}\ e{\isadigit{1}}\ {\isasymnoteq}\ e{\isadigit{2}}{\isacharparenright}{\kern0pt}{\isachardoublequoteclose}\ \isakeywordTWO{and}\isanewline
\ \ \ \ ST{\isadigit{3}}{\isacharunderscore}{\kern0pt}curvature{\isacharunderscore}{\kern0pt}emergent{\isacharcolon}{\kern0pt}\isanewline
\ \ \ \ \ \ {\isachardoublequoteopen}{\isasymforall}f\ e{\isachardot}{\kern0pt}\ coord\ f\ e\ {\isasymnoteq}\ None\ {\isasymlongrightarrow}\isanewline
\ \ \ \ \ \ \ \ \ {\isacharparenleft}{\kern0pt}{\isasymexists}es{\isachardot}{\kern0pt}\ {\isacharparenleft}{\kern0pt}{\isasymforall}e{\isacharprime}{\kern0pt}\ {\isasymin}\ es{\isachardot}{\kern0pt}\ Phenomenon\ e{\isacharprime}{\kern0pt}{\isacharparenright}{\kern0pt}\ {\isasymand}\ e\ {\isasymin}\ es{\isacharparenright}{\kern0pt}{\isachardoublequoteclose}\isanewline
\isanewline
\ \ \isakeywordONE{lemma}\isamarkupfalse%
\ Gravity{\isacharunderscore}{\kern0pt}relational{\isacharunderscore}{\kern0pt}presentation{\isacharcolon}{\kern0pt}\isanewline
\ \ \ \ \isakeywordTWO{assumes}\ {\isachardoublequoteopen}GravField\ e{\isachardoublequoteclose}\isanewline
\ \ \ \ \isakeywordTWO{shows}\ {\isachardoublequoteopen}Inseparable\ e\ {\isasymOmega}{\isachardoublequoteclose}\isanewline
%
\isadelimproof
\ \ \ \ %
\endisadelimproof
%
\isatagproof
\isakeywordONE{using}\isamarkupfalse%
\ assms\ ST{\isadigit{2}}{\isacharunderscore}{\kern0pt}gravity{\isacharunderscore}{\kern0pt}relational\ Nonduality\ \isakeywordONE{by}\isamarkupfalse%
\ blast%
\endisatagproof
{\isafoldproof}%
%
\isadelimproof
\isanewline
%
\endisadelimproof
\isanewline
\ \ \isanewline
\ \ \isakeywordONE{theorem}\isamarkupfalse%
\ Spacetime{\isacharunderscore}{\kern0pt}emerges{\isacharunderscore}{\kern0pt}from{\isacharunderscore}{\kern0pt}presentations{\isacharcolon}{\kern0pt}\isanewline
\ \ \ \ {\isachardoublequoteopen}{\isasymforall}f\ e\ r{\isachardot}{\kern0pt}\ coord\ f\ e\ {\isacharequal}{\kern0pt}\ Some\ r\ {\isasymlongrightarrow}\ Inseparable\ e\ {\isasymOmega}{\isachardoublequoteclose}\isanewline
%
\isadelimproof
\ \ %
\endisadelimproof
%
\isatagproof
\isakeywordONE{proof}\isamarkupfalse%
\ {\isacharparenleft}{\kern0pt}intro\ allI\ impI{\isacharparenright}{\kern0pt}\isanewline
\ \ \ \ \isakeywordTHREE{fix}\isamarkupfalse%
\ f\ e\ r\isanewline
\ \ \ \ \isakeywordTHREE{assume}\isamarkupfalse%
\ H{\isacharcolon}{\kern0pt}\ {\isachardoublequoteopen}coord\ f\ e\ {\isacharequal}{\kern0pt}\ Some\ r{\isachardoublequoteclose}\isanewline
\ \ \ \ \isakeywordONE{hence}\isamarkupfalse%
\ {\isachardoublequoteopen}coord\ f\ e\ {\isasymnoteq}\ None{\isachardoublequoteclose}\ \isakeywordONE{by}\isamarkupfalse%
\ simp\isanewline
\ \ \ \ \isakeywordONE{then}\isamarkupfalse%
\ \isakeywordTHREE{obtain}\isamarkupfalse%
\ es\ \isakeywordTWO{where}\ Es{\isacharcolon}{\kern0pt}\ {\isachardoublequoteopen}{\isacharparenleft}{\kern0pt}{\isasymforall}e{\isacharprime}{\kern0pt}\ {\isasymin}\ es{\isachardot}{\kern0pt}\ Phenomenon\ e{\isacharprime}{\kern0pt}{\isacharparenright}{\kern0pt}\ {\isasymand}\ e\ {\isasymin}\ es{\isachardoublequoteclose}\isanewline
\ \ \ \ \ \ \isakeywordONE{using}\isamarkupfalse%
\ ST{\isadigit{3}}{\isacharunderscore}{\kern0pt}curvature{\isacharunderscore}{\kern0pt}emergent\ \isakeywordONE{by}\isamarkupfalse%
\ blast\isanewline
\ \ \ \ \isakeywordONE{hence}\isamarkupfalse%
\ {\isachardoublequoteopen}Phenomenon\ e{\isachardoublequoteclose}\ \isakeywordONE{by}\isamarkupfalse%
\ blast\isanewline
\ \ \ \ \isakeywordTHREE{thus}\isamarkupfalse%
\ {\isachardoublequoteopen}Inseparable\ e\ {\isasymOmega}{\isachardoublequoteclose}\ \isakeywordONE{using}\isamarkupfalse%
\ Nonduality\ \isakeywordONE{by}\isamarkupfalse%
\ blast\isanewline
\ \ \isakeywordONE{qed}\isamarkupfalse%
%
\endisatagproof
{\isafoldproof}%
%
\isadelimproof
%
\endisadelimproof
%
\isadelimdocument
%
\endisadelimdocument
%
\isatagdocument
%
\isamarkupsection{Presentation Dynamics and Field Equations%
}
\isamarkuptrue%
%
\endisatagdocument
{\isafolddocument}%
%
\isadelimdocument
%
\endisadelimdocument
\ \ \isakeywordONE{consts}\isamarkupfalse%
\isanewline
\ \ \ \ PresConsistent\ {\isacharcolon}{\kern0pt}{\isacharcolon}{\kern0pt}\ {\isachardoublequoteopen}E\ set\ {\isasymRightarrow}\ bool{\isachardoublequoteclose}\ \ \ \isanewline
\ \ \ \ PresEvolves\ \ \ \ {\isacharcolon}{\kern0pt}{\isacharcolon}{\kern0pt}\ {\isachardoublequoteopen}E\ {\isasymRightarrow}\ E\ {\isasymRightarrow}\ bool{\isachardoublequoteclose}\ \ \isanewline
\isanewline
\ \ \isakeywordONE{axiomatization}\isamarkupfalse%
\ \isakeywordTWO{where}\isanewline
\ \ \ \ PD{\isadigit{1}}{\isacharunderscore}{\kern0pt}consistency{\isacharunderscore}{\kern0pt}requires{\isacharunderscore}{\kern0pt}unity{\isacharcolon}{\kern0pt}\isanewline
\ \ \ \ \ \ {\isachardoublequoteopen}{\isasymforall}es{\isachardot}{\kern0pt}\ PresConsistent\ es\ {\isasymlongrightarrow}\ {\isacharparenleft}{\kern0pt}{\isasymforall}e\ {\isasymin}\ es{\isachardot}{\kern0pt}\ Phenomenon\ e\ {\isasymand}\ Presents\ e\ {\isasymOmega}{\isacharparenright}{\kern0pt}{\isachardoublequoteclose}\ \isakeywordTWO{and}\isanewline
\ \ \ \ PD{\isadigit{2}}{\isacharunderscore}{\kern0pt}evolution{\isacharunderscore}{\kern0pt}causal{\isacharcolon}{\kern0pt}\isanewline
\ \ \ \ \ \ {\isachardoublequoteopen}{\isasymforall}e{\isadigit{1}}\ e{\isadigit{2}}{\isachardot}{\kern0pt}\ PresEvolves\ e{\isadigit{1}}\ e{\isadigit{2}}\ {\isasymlongrightarrow}\isanewline
\ \ \ \ \ \ \ \ \ Phenomenon\ e{\isadigit{1}}\ {\isasymand}\ Phenomenon\ e{\isadigit{2}}\ {\isasymand}\ CausallyPrecedes\ e{\isadigit{1}}\ e{\isadigit{2}}{\isachardoublequoteclose}\ \isakeywordTWO{and}\isanewline
\ \ \ \ PD{\isadigit{3}}{\isacharunderscore}{\kern0pt}evolution{\isacharunderscore}{\kern0pt}preserves{\isacharunderscore}{\kern0pt}substrate{\isacharcolon}{\kern0pt}\isanewline
\ \ \ \ \ \ {\isachardoublequoteopen}{\isasymforall}e{\isadigit{1}}\ e{\isadigit{2}}{\isachardot}{\kern0pt}\ PresEvolves\ e{\isadigit{1}}\ e{\isadigit{2}}\ {\isasymlongrightarrow}\ Presents\ e{\isadigit{1}}\ {\isasymOmega}\ {\isasymand}\ Presents\ e{\isadigit{2}}\ {\isasymOmega}{\isachardoublequoteclose}\isanewline
\isanewline
\ \ \isakeywordONE{lemma}\isamarkupfalse%
\ Presentation{\isacharunderscore}{\kern0pt}evolution{\isacharunderscore}{\kern0pt}nondual{\isacharcolon}{\kern0pt}\isanewline
\ \ \ \ \isakeywordTWO{assumes}\ {\isachardoublequoteopen}PresEvolves\ e{\isadigit{1}}\ e{\isadigit{2}}{\isachardoublequoteclose}\isanewline
\ \ \ \ \isakeywordTWO{shows}\ {\isachardoublequoteopen}Inseparable\ e{\isadigit{1}}\ {\isasymOmega}\ {\isasymand}\ Inseparable\ e{\isadigit{2}}\ {\isasymOmega}{\isachardoublequoteclose}\isanewline
%
\isadelimproof
\ \ \ \ %
\endisadelimproof
%
\isatagproof
\isakeywordONE{using}\isamarkupfalse%
\ assms\ PD{\isadigit{2}}{\isacharunderscore}{\kern0pt}evolution{\isacharunderscore}{\kern0pt}causal\ Nonduality\ \isakeywordONE{by}\isamarkupfalse%
\ blast%
\endisatagproof
{\isafoldproof}%
%
\isadelimproof
\isanewline
%
\endisadelimproof
\isanewline
\ \ \isakeywordONE{theorem}\isamarkupfalse%
\ Consistent{\isacharunderscore}{\kern0pt}copresentation{\isacharunderscore}{\kern0pt}unified{\isacharcolon}{\kern0pt}\isanewline
\ \ \ \ {\isachardoublequoteopen}{\isasymforall}es{\isachardot}{\kern0pt}\ PresConsistent\ es\ {\isasymlongrightarrow}\ {\isacharparenleft}{\kern0pt}{\isasymforall}e\ {\isasymin}\ es{\isachardot}{\kern0pt}\ Inseparable\ e\ {\isasymOmega}{\isacharparenright}{\kern0pt}{\isachardoublequoteclose}\isanewline
%
\isadelimproof
\ \ %
\endisadelimproof
%
\isatagproof
\isakeywordONE{proof}\isamarkupfalse%
\ {\isacharparenleft}{\kern0pt}intro\ allI\ impI\ ballI{\isacharparenright}{\kern0pt}\isanewline
\ \ \ \ \isakeywordTHREE{fix}\isamarkupfalse%
\ es\ e\isanewline
\ \ \ \ \isakeywordTHREE{assume}\isamarkupfalse%
\ {\isachardoublequoteopen}PresConsistent\ es{\isachardoublequoteclose}\ \isakeywordTWO{and}\ {\isachardoublequoteopen}e\ {\isasymin}\ es{\isachardoublequoteclose}\isanewline
\ \ \ \ \isakeywordONE{hence}\isamarkupfalse%
\ {\isachardoublequoteopen}Phenomenon\ e\ {\isasymand}\ Presents\ e\ {\isasymOmega}{\isachardoublequoteclose}\isanewline
\ \ \ \ \ \ \isakeywordONE{using}\isamarkupfalse%
\ PD{\isadigit{1}}{\isacharunderscore}{\kern0pt}consistency{\isacharunderscore}{\kern0pt}requires{\isacharunderscore}{\kern0pt}unity\ \isakeywordONE{by}\isamarkupfalse%
\ blast\isanewline
\ \ \ \ \isakeywordTHREE{thus}\isamarkupfalse%
\ {\isachardoublequoteopen}Inseparable\ e\ {\isasymOmega}{\isachardoublequoteclose}\ \isakeywordONE{using}\isamarkupfalse%
\ Nonduality\ \isakeywordONE{by}\isamarkupfalse%
\ blast\isanewline
\ \ \isakeywordONE{qed}\isamarkupfalse%
%
\endisatagproof
{\isafoldproof}%
%
\isadelimproof
%
\endisadelimproof
%
\isadelimdocument
%
\endisadelimdocument
%
\isatagdocument
%
\isamarkupsection{Master Unification Theorem%
}
\isamarkuptrue%
%
\endisatagdocument
{\isafolddocument}%
%
\isadelimdocument
%
\endisadelimdocument
\ \ \isakeywordONE{theorem}\isamarkupfalse%
\ Ontological{\isacharunderscore}{\kern0pt}Unification{\isacharcolon}{\kern0pt}\isanewline
\ \ \ \ {\isachardoublequoteopen}{\isasymforall}e{\isachardot}{\kern0pt}\ Phenomenon\ e\ {\isasymlongrightarrow}\isanewline
\ \ \ \ \ \ {\isacharparenleft}{\kern0pt}{\isasymforall}ft\ gg\ force{\isachardot}{\kern0pt}\isanewline
\ \ \ \ \ \ \ \ {\isacharparenleft}{\kern0pt}Excitation\ e\ ft\ {\isasymlongrightarrow}\ Inseparable\ e\ {\isasymOmega}{\isacharparenright}{\kern0pt}\ {\isasymand}\isanewline
\ \ \ \ \ \ \ \ {\isacharparenleft}{\kern0pt}IndexScheme\ gg\ e\ {\isasymlongrightarrow}\ Inseparable\ e\ {\isasymOmega}{\isacharparenright}{\kern0pt}\ {\isasymand}\isanewline
\ \ \ \ \ \ \ \ {\isacharparenleft}{\kern0pt}ForcePresentation\ e\ force\ {\isasymlongrightarrow}\ Inseparable\ e\ {\isasymOmega}{\isacharparenright}{\kern0pt}\ {\isasymand}\isanewline
\ \ \ \ \ \ \ \ {\isacharparenleft}{\kern0pt}{\isasymforall}e{\isadigit{2}}{\isachardot}{\kern0pt}\ Entangled\ e\ e{\isadigit{2}}\ {\isasymlongrightarrow}\ Inseparable\ e\ {\isasymOmega}\ {\isasymand}\ Inseparable\ e{\isadigit{2}}\ {\isasymOmega}{\isacharparenright}{\kern0pt}{\isacharparenright}{\kern0pt}{\isachardoublequoteclose}\isanewline
%
\isadelimproof
\ \ %
\endisadelimproof
%
\isatagproof
\isakeywordONE{proof}\isamarkupfalse%
\ {\isacharparenleft}{\kern0pt}intro\ allI\ impI{\isacharparenright}{\kern0pt}\isanewline
\ \ \ \ \isakeywordTHREE{fix}\isamarkupfalse%
\ e\ ft\ gg\ force\isanewline
\ \ \ \ \isakeywordTHREE{assume}\isamarkupfalse%
\ P{\isacharcolon}{\kern0pt}\ {\isachardoublequoteopen}Phenomenon\ e{\isachardoublequoteclose}\isanewline
\ \ \ \ \isakeywordTHREE{show}\isamarkupfalse%
\ {\isachardoublequoteopen}{\isacharparenleft}{\kern0pt}Excitation\ e\ ft\ {\isasymlongrightarrow}\ Inseparable\ e\ {\isasymOmega}{\isacharparenright}{\kern0pt}\ {\isasymand}\isanewline
\ \ \ \ \ \ \ \ \ \ {\isacharparenleft}{\kern0pt}IndexScheme\ gg\ e\ {\isasymlongrightarrow}\ Inseparable\ e\ {\isasymOmega}{\isacharparenright}{\kern0pt}\ {\isasymand}\isanewline
\ \ \ \ \ \ \ \ \ \ {\isacharparenleft}{\kern0pt}ForcePresentation\ e\ force\ {\isasymlongrightarrow}\ Inseparable\ e\ {\isasymOmega}{\isacharparenright}{\kern0pt}\ {\isasymand}\isanewline
\ \ \ \ \ \ \ \ \ \ {\isacharparenleft}{\kern0pt}{\isasymforall}e{\isadigit{2}}{\isachardot}{\kern0pt}\ Entangled\ e\ e{\isadigit{2}}\ {\isasymlongrightarrow}\ Inseparable\ e\ {\isasymOmega}\ {\isasymand}\ Inseparable\ e{\isadigit{2}}\ {\isasymOmega}{\isacharparenright}{\kern0pt}{\isachardoublequoteclose}\isanewline
\ \ \ \ \isakeywordONE{proof}\isamarkupfalse%
\ {\isacharparenleft}{\kern0pt}intro\ conjI\ impI\ allI{\isacharparenright}{\kern0pt}\isanewline
\ \ \ \ \ \ \isakeywordTHREE{assume}\isamarkupfalse%
\ {\isachardoublequoteopen}Excitation\ e\ ft{\isachardoublequoteclose}\isanewline
\ \ \ \ \ \ \isakeywordTHREE{thus}\isamarkupfalse%
\ {\isachardoublequoteopen}Inseparable\ e\ {\isasymOmega}{\isachardoublequoteclose}\ \isakeywordONE{using}\isamarkupfalse%
\ Field{\isacharunderscore}{\kern0pt}channels{\isacharunderscore}{\kern0pt}structure{\isacharunderscore}{\kern0pt}presentation\ \isakeywordONE{by}\isamarkupfalse%
\ blast\isanewline
\ \ \ \ \isakeywordONE{next}\isamarkupfalse%
\isanewline
\ \ \ \ \ \ \isakeywordTHREE{assume}\isamarkupfalse%
\ {\isachardoublequoteopen}IndexScheme\ gg\ e{\isachardoublequoteclose}\isanewline
\ \ \ \ \ \ \isakeywordTHREE{thus}\isamarkupfalse%
\ {\isachardoublequoteopen}Inseparable\ e\ {\isasymOmega}{\isachardoublequoteclose}\ \isakeywordONE{using}\isamarkupfalse%
\ Gauge{\isacharunderscore}{\kern0pt}indexing{\isacharunderscore}{\kern0pt}preserves{\isacharunderscore}{\kern0pt}nonduality\ \isakeywordONE{by}\isamarkupfalse%
\ blast\isanewline
\ \ \ \ \isakeywordONE{next}\isamarkupfalse%
\isanewline
\ \ \ \ \ \ \isakeywordTHREE{assume}\isamarkupfalse%
\ {\isachardoublequoteopen}ForcePresentation\ e\ force{\isachardoublequoteclose}\isanewline
\ \ \ \ \ \ \isakeywordTHREE{thus}\isamarkupfalse%
\ {\isachardoublequoteopen}Inseparable\ e\ {\isasymOmega}{\isachardoublequoteclose}\ \isakeywordONE{using}\isamarkupfalse%
\ Force{\isacharunderscore}{\kern0pt}phenomena{\isacharunderscore}{\kern0pt}nondual\ \isakeywordONE{by}\isamarkupfalse%
\ blast\isanewline
\ \ \ \ \isakeywordONE{next}\isamarkupfalse%
\isanewline
\ \ \ \ \ \ \isakeywordTHREE{fix}\isamarkupfalse%
\ e{\isadigit{2}}\isanewline
\ \ \ \ \ \ \isakeywordTHREE{assume}\isamarkupfalse%
\ {\isachardoublequoteopen}Entangled\ e\ e{\isadigit{2}}{\isachardoublequoteclose}\isanewline
\ \ \ \ \ \ \isakeywordONE{then}\isamarkupfalse%
\ \isakeywordTHREE{show}\isamarkupfalse%
\ {\isachardoublequoteopen}Inseparable\ e\ {\isasymOmega}{\isachardoublequoteclose}\ \isakeywordONE{using}\isamarkupfalse%
\ Entanglement{\isacharunderscore}{\kern0pt}via{\isacharunderscore}{\kern0pt}nonduality\ \isakeywordONE{by}\isamarkupfalse%
\ blast\isanewline
\ \ \ \ \isakeywordONE{next}\isamarkupfalse%
\isanewline
\ \ \ \ \ \ \isakeywordTHREE{fix}\isamarkupfalse%
\ e{\isadigit{2}}\isanewline
\ \ \ \ \ \ \isakeywordTHREE{assume}\isamarkupfalse%
\ {\isachardoublequoteopen}Entangled\ e\ e{\isadigit{2}}{\isachardoublequoteclose}\isanewline
\ \ \ \ \ \ \isakeywordONE{then}\isamarkupfalse%
\ \isakeywordTHREE{show}\isamarkupfalse%
\ {\isachardoublequoteopen}Inseparable\ e{\isadigit{2}}\ {\isasymOmega}{\isachardoublequoteclose}\ \isakeywordONE{using}\isamarkupfalse%
\ Entanglement{\isacharunderscore}{\kern0pt}via{\isacharunderscore}{\kern0pt}nonduality\ \isakeywordONE{by}\isamarkupfalse%
\ blast\isanewline
\ \ \ \ \isakeywordONE{qed}\isamarkupfalse%
\isanewline
\ \ \isakeywordONE{qed}\isamarkupfalse%
%
\endisatagproof
{\isafoldproof}%
%
\isadelimproof
\isanewline
%
\endisadelimproof
\isanewline
\ \ \isakeywordONE{corollary}\isamarkupfalse%
\ All{\isacharunderscore}{\kern0pt}physical{\isacharunderscore}{\kern0pt}phenomena{\isacharunderscore}{\kern0pt}unified{\isacharcolon}{\kern0pt}\isanewline
\ \ \ \ {\isachardoublequoteopen}{\isasymforall}e{\isadigit{1}}\ e{\isadigit{2}}{\isachardot}{\kern0pt}\ Phenomenon\ e{\isadigit{1}}\ {\isasymand}\ Phenomenon\ e{\isadigit{2}}\ {\isasymlongrightarrow}\isanewline
\ \ \ \ \ \ \ {\isacharparenleft}{\kern0pt}{\isasymexists}s{\isachardot}{\kern0pt}\ Substrate\ s\ {\isasymand}\ Presents\ e{\isadigit{1}}\ s\ {\isasymand}\ Presents\ e{\isadigit{2}}\ s\ {\isasymand}\ s\ {\isacharequal}{\kern0pt}\ {\isasymOmega}{\isacharparenright}{\kern0pt}{\isachardoublequoteclose}\isanewline
%
\isadelimproof
\ \ \ \ %
\endisadelimproof
%
\isatagproof
\isakeywordONE{using}\isamarkupfalse%
\ substrate{\isacharunderscore}{\kern0pt}Omega\ A{\isadigit{4}}{\isacharunderscore}{\kern0pt}presentation\ \isakeywordONE{by}\isamarkupfalse%
\ blast%
\endisatagproof
{\isafoldproof}%
%
\isadelimproof
\isanewline
%
\endisadelimproof
\isanewline
\ \ \isakeywordONE{corollary}\isamarkupfalse%
\ Forces{\isacharunderscore}{\kern0pt}unified{\isacharunderscore}{\kern0pt}via{\isacharunderscore}{\kern0pt}substrate{\isacharcolon}{\kern0pt}\isanewline
\ \ \ \ {\isachardoublequoteopen}{\isasymforall}e{\isadigit{1}}\ e{\isadigit{2}}\ f{\isadigit{1}}\ f{\isadigit{2}}{\isachardot}{\kern0pt}\ ForcePresentation\ e{\isadigit{1}}\ f{\isadigit{1}}\ {\isasymand}\ ForcePresentation\ e{\isadigit{2}}\ f{\isadigit{2}}\ {\isasymlongrightarrow}\isanewline
\ \ \ \ \ \ \ Inseparable\ e{\isadigit{1}}\ {\isasymOmega}\ {\isasymand}\ Inseparable\ e{\isadigit{2}}\ {\isasymOmega}{\isachardoublequoteclose}\isanewline
%
\isadelimproof
\ \ \ \ %
\endisadelimproof
%
\isatagproof
\isakeywordONE{using}\isamarkupfalse%
\ Force{\isacharunderscore}{\kern0pt}phenomena{\isacharunderscore}{\kern0pt}nondual\ \isakeywordONE{by}\isamarkupfalse%
\ blast%
\endisatagproof
{\isafoldproof}%
%
\isadelimproof
%
\endisadelimproof
%
\isadelimdocument
%
\endisadelimdocument
%
\isatagdocument
%
\isamarkupsection{Testable Predictions Framework%
}
\isamarkuptrue%
%
\endisatagdocument
{\isafolddocument}%
%
\isadelimdocument
%
\endisadelimdocument
\ \ \isakeywordONE{consts}\isamarkupfalse%
\isanewline
\ \ \ \ SubstrateMediatedCorr\ {\isacharcolon}{\kern0pt}{\isacharcolon}{\kern0pt}\ {\isachardoublequoteopen}E\ {\isasymRightarrow}\ E\ {\isasymRightarrow}\ bool{\isachardoublequoteclose}\ \ \isanewline
\ \ \ \ LocalInteraction\ \ \ \ \ \ {\isacharcolon}{\kern0pt}{\isacharcolon}{\kern0pt}\ {\isachardoublequoteopen}E\ {\isasymRightarrow}\ E\ {\isasymRightarrow}\ bool{\isachardoublequoteclose}\ \ \isanewline
\isanewline
\ \ \isakeywordONE{axiomatization}\isamarkupfalse%
\ \isakeywordTWO{where}\isanewline
\ \ \ \ TEST{\isadigit{1}}{\isacharunderscore}{\kern0pt}substrate{\isacharunderscore}{\kern0pt}vs{\isacharunderscore}{\kern0pt}local{\isacharcolon}{\kern0pt}\isanewline
\ \ \ \ \ \ {\isachardoublequoteopen}{\isasymforall}e{\isadigit{1}}\ e{\isadigit{2}}{\isachardot}{\kern0pt}\ SubstrateMediatedCorr\ e{\isadigit{1}}\ e{\isadigit{2}}\ {\isasymlongrightarrow}\isanewline
\ \ \ \ \ \ \ \ \ Phenomenon\ e{\isadigit{1}}\ {\isasymand}\ Phenomenon\ e{\isadigit{2}}\ {\isasymand}\ {\isasymnot}\ LocalInteraction\ e{\isadigit{1}}\ e{\isadigit{2}}{\isachardoublequoteclose}\ \isakeywordTWO{and}\isanewline
\ \ \ \ TEST{\isadigit{2}}{\isacharunderscore}{\kern0pt}no{\isacharunderscore}{\kern0pt}superluminal{\isacharunderscore}{\kern0pt}causation{\isacharcolon}{\kern0pt}\isanewline
\ \ \ \ \ \ {\isachardoublequoteopen}{\isasymforall}e{\isadigit{1}}\ e{\isadigit{2}}{\isachardot}{\kern0pt}\ CausallyPrecedes\ e{\isadigit{1}}\ e{\isadigit{2}}\ {\isasymlongrightarrow}\ {\isasymnot}\ SubstrateMediatedCorr\ e{\isadigit{1}}\ e{\isadigit{2}}{\isachardoublequoteclose}\ \isakeywordTWO{and}\isanewline
\ \ \ \ TEST{\isadigit{3}}{\isacharunderscore}{\kern0pt}substrate{\isacharunderscore}{\kern0pt}corr{\isacharunderscore}{\kern0pt}symmetric{\isacharcolon}{\kern0pt}\isanewline
\ \ \ \ \ \ {\isachardoublequoteopen}{\isasymforall}e{\isadigit{1}}\ e{\isadigit{2}}{\isachardot}{\kern0pt}\ SubstrateMediatedCorr\ e{\isadigit{1}}\ e{\isadigit{2}}\ {\isasymlongleftrightarrow}\ SubstrateMediatedCorr\ e{\isadigit{2}}\ e{\isadigit{1}}{\isachardoublequoteclose}\isanewline
\isanewline
\ \ \isakeywordONE{lemma}\isamarkupfalse%
\ Substrate{\isacharunderscore}{\kern0pt}correlations{\isacharunderscore}{\kern0pt}acausal{\isacharcolon}{\kern0pt}\isanewline
\ \ \ \ \isakeywordTWO{assumes}\ {\isachardoublequoteopen}SubstrateMediatedCorr\ e{\isadigit{1}}\ e{\isadigit{2}}{\isachardoublequoteclose}\isanewline
\ \ \ \ \isakeywordTWO{shows}\ {\isachardoublequoteopen}{\isasymnot}\ CausallyPrecedes\ e{\isadigit{1}}\ e{\isadigit{2}}\ {\isasymand}\ {\isasymnot}\ CausallyPrecedes\ e{\isadigit{2}}\ e{\isadigit{1}}{\isachardoublequoteclose}\isanewline
%
\isadelimproof
\ \ %
\endisadelimproof
%
\isatagproof
\isakeywordONE{proof}\isamarkupfalse%
\ {\isacharparenleft}{\kern0pt}intro\ conjI{\isacharparenright}{\kern0pt}\isanewline
\ \ \ \ \isakeywordTHREE{show}\isamarkupfalse%
\ {\isachardoublequoteopen}{\isasymnot}\ CausallyPrecedes\ e{\isadigit{1}}\ e{\isadigit{2}}{\isachardoublequoteclose}\isanewline
\ \ \ \ \ \ \isakeywordONE{using}\isamarkupfalse%
\ assms\ TEST{\isadigit{2}}{\isacharunderscore}{\kern0pt}no{\isacharunderscore}{\kern0pt}superluminal{\isacharunderscore}{\kern0pt}causation\ \isakeywordONE{by}\isamarkupfalse%
\ blast\isanewline
\ \ \isakeywordONE{next}\isamarkupfalse%
\isanewline
\ \ \ \ \isakeywordTHREE{show}\isamarkupfalse%
\ {\isachardoublequoteopen}{\isasymnot}\ CausallyPrecedes\ e{\isadigit{2}}\ e{\isadigit{1}}{\isachardoublequoteclose}\isanewline
\ \ \ \ \ \ \isakeywordONE{using}\isamarkupfalse%
\ assms\ TEST{\isadigit{2}}{\isacharunderscore}{\kern0pt}no{\isacharunderscore}{\kern0pt}superluminal{\isacharunderscore}{\kern0pt}causation\ TEST{\isadigit{3}}{\isacharunderscore}{\kern0pt}substrate{\isacharunderscore}{\kern0pt}corr{\isacharunderscore}{\kern0pt}symmetric\ \isakeywordONE{by}\isamarkupfalse%
\ blast\isanewline
\ \ \isakeywordONE{qed}\isamarkupfalse%
%
\endisatagproof
{\isafoldproof}%
%
\isadelimproof
\isanewline
%
\endisadelimproof
\isanewline
\ \ \isakeywordONE{theorem}\isamarkupfalse%
\ Distinguishable{\isacharunderscore}{\kern0pt}from{\isacharunderscore}{\kern0pt}local{\isacharunderscore}{\kern0pt}hidden{\isacharunderscore}{\kern0pt}variables{\isacharcolon}{\kern0pt}\isanewline
\ \ \ \ {\isachardoublequoteopen}{\isasymforall}e{\isadigit{1}}\ e{\isadigit{2}}{\isachardot}{\kern0pt}\ SubstrateMediatedCorr\ e{\isadigit{1}}\ e{\isadigit{2}}\ {\isasymlongrightarrow}\isanewline
\ \ \ \ \ \ \ Inseparable\ e{\isadigit{1}}\ {\isasymOmega}\ {\isasymand}\ Inseparable\ e{\isadigit{2}}\ {\isasymOmega}\ {\isasymand}\isanewline
\ \ \ \ \ \ \ {\isasymnot}\ CausallyPrecedes\ e{\isadigit{1}}\ e{\isadigit{2}}\ {\isasymand}\ {\isasymnot}\ CausallyPrecedes\ e{\isadigit{2}}\ e{\isadigit{1}}{\isachardoublequoteclose}\isanewline
%
\isadelimproof
\ \ \ \ %
\endisadelimproof
%
\isatagproof
\isakeywordONE{using}\isamarkupfalse%
\ TEST{\isadigit{1}}{\isacharunderscore}{\kern0pt}substrate{\isacharunderscore}{\kern0pt}vs{\isacharunderscore}{\kern0pt}local\ Nonduality\ Substrate{\isacharunderscore}{\kern0pt}correlations{\isacharunderscore}{\kern0pt}acausal\ \isakeywordONE{by}\isamarkupfalse%
\ blast%
\endisatagproof
{\isafoldproof}%
%
\isadelimproof
%
\endisadelimproof
%
\isadelimdocument
%
\endisadelimdocument
%
\isatagdocument
%
\isamarkupsection{Implementation Roadmap Formalization%
}
\isamarkuptrue%
%
\endisatagdocument
{\isafolddocument}%
%
\isadelimdocument
%
\endisadelimdocument
\ \ \isakeywordONE{consts}\isamarkupfalse%
\isanewline
\ \ \ \ TypeMapping\ \ \ \ \ {\isacharcolon}{\kern0pt}{\isacharcolon}{\kern0pt}\ {\isachardoublequoteopen}Label\ {\isasymRightarrow}\ bool{\isachardoublequoteclose}\ \ \isanewline
\ \ \ \ OperationalCrit\ {\isacharcolon}{\kern0pt}{\isacharcolon}{\kern0pt}\ {\isachardoublequoteopen}E\ {\isasymRightarrow}\ bool{\isachardoublequoteclose}\ \ \ \ \ \ \isanewline
\ \ \ \ EmergentTheory\ \ {\isacharcolon}{\kern0pt}{\isacharcolon}{\kern0pt}\ {\isachardoublequoteopen}Label\ {\isasymRightarrow}\ bool{\isachardoublequoteclose}\ \ \isanewline
\ \ \ \ LabelE\ \ \ \ \ \ \ \ \ \ {\isacharcolon}{\kern0pt}{\isacharcolon}{\kern0pt}\ {\isachardoublequoteopen}Label{\isachardoublequoteclose}\ \ \isanewline
\ \ \ \ LabelFrame\ \ \ \ \ \ {\isacharcolon}{\kern0pt}{\isacharcolon}{\kern0pt}\ {\isachardoublequoteopen}Label{\isachardoublequoteclose}\ \ \isanewline
\ \ \ \ LabelG\ \ \ \ \ \ \ \ \ \ {\isacharcolon}{\kern0pt}{\isacharcolon}{\kern0pt}\ {\isachardoublequoteopen}Label{\isachardoublequoteclose}\ \ \isanewline
\ \ \ \ LabelQ\ \ \ \ \ \ \ \ \ \ {\isacharcolon}{\kern0pt}{\isacharcolon}{\kern0pt}\ {\isachardoublequoteopen}Label{\isachardoublequoteclose}\ \ \isanewline
\ \ \ \ LabelSM\ \ \ \ \ \ \ \ \ {\isacharcolon}{\kern0pt}{\isacharcolon}{\kern0pt}\ {\isachardoublequoteopen}Label{\isachardoublequoteclose}\ \ \isanewline
\ \ \ \ LabelGR\ \ \ \ \ \ \ \ \ {\isacharcolon}{\kern0pt}{\isacharcolon}{\kern0pt}\ {\isachardoublequoteopen}Label{\isachardoublequoteclose}\ \ \isanewline
\isanewline
\ \ \isakeywordONE{axiomatization}\isamarkupfalse%
\ \isakeywordTWO{where}\isanewline
\ \ \ \ IMP{\isadigit{1}}{\isacharunderscore}{\kern0pt}type{\isacharunderscore}{\kern0pt}completeness{\isacharcolon}{\kern0pt}\isanewline
\ \ \ \ \ \ {\isachardoublequoteopen}TypeMapping\ LabelE\ {\isasymand}\ TypeMapping\ LabelFrame\ {\isasymand}\ TypeMapping\ LabelG\ {\isasymand}\ TypeMapping\ LabelQ{\isachardoublequoteclose}\ \isakeywordTWO{and}\isanewline
\ \ \ \ IMP{\isadigit{2}}{\isacharunderscore}{\kern0pt}operational{\isacharunderscore}{\kern0pt}defined{\isacharcolon}{\kern0pt}\isanewline
\ \ \ \ \ \ {\isachardoublequoteopen}{\isasymforall}e{\isachardot}{\kern0pt}\ Phenomenon\ e\ {\isasymlongrightarrow}\ OperationalCrit\ e{\isachardoublequoteclose}\ \isakeywordTWO{and}\isanewline
\ \ \ \ IMP{\isadigit{3}}{\isacharunderscore}{\kern0pt}standard{\isacharunderscore}{\kern0pt}model{\isacharunderscore}{\kern0pt}emergent{\isacharcolon}{\kern0pt}\isanewline
\ \ \ \ \ \ {\isachardoublequoteopen}EmergentTheory\ LabelSM\ {\isasymand}\ EmergentTheory\ LabelGR{\isachardoublequoteclose}\isanewline
\isanewline
\ \ \isakeywordONE{lemma}\isamarkupfalse%
\ Implementation{\isacharunderscore}{\kern0pt}preserves{\isacharunderscore}{\kern0pt}nonduality{\isacharcolon}{\kern0pt}\isanewline
\ \ \ \ \isakeywordTWO{assumes}\ {\isachardoublequoteopen}OperationalCrit\ e{\isachardoublequoteclose}\ \isakeywordTWO{and}\ {\isachardoublequoteopen}Phenomenon\ e{\isachardoublequoteclose}\isanewline
\ \ \ \ \isakeywordTWO{shows}\ {\isachardoublequoteopen}Inseparable\ e\ {\isasymOmega}{\isachardoublequoteclose}\isanewline
%
\isadelimproof
\ \ \ \ %
\endisadelimproof
%
\isatagproof
\isakeywordONE{using}\isamarkupfalse%
\ assms\ Nonduality\ \isakeywordONE{by}\isamarkupfalse%
\ blast%
\endisatagproof
{\isafoldproof}%
%
\isadelimproof
%
\endisadelimproof
%
\isadelimdocument
%
\endisadelimdocument
%
\isatagdocument
%
\isamarkupsection{Consistency and Completeness%
}
\isamarkuptrue%
%
\endisatagdocument
{\isafolddocument}%
%
\isadelimdocument
%
\endisadelimdocument
\ \ \isakeywordONE{lemma}\isamarkupfalse%
\ Unified{\isacharunderscore}{\kern0pt}theory{\isacharunderscore}{\kern0pt}consistent{\isacharcolon}{\kern0pt}\ True%
\isadelimproof
\ %
\endisadelimproof
%
\isatagproof
\isakeywordONE{by}\isamarkupfalse%
\ simp%
\endisatagproof
{\isafoldproof}%
%
\isadelimproof
%
\endisadelimproof
\isanewline
\isanewline
\ \ \isakeywordONE{theorem}\isamarkupfalse%
\ Framework{\isacharunderscore}{\kern0pt}accommodates{\isacharunderscore}{\kern0pt}all{\isacharunderscore}{\kern0pt}forces{\isacharcolon}{\kern0pt}\isanewline
\ \ \ \ {\isachardoublequoteopen}{\isasymforall}force{\isachardot}{\kern0pt}\ {\isacharparenleft}{\kern0pt}{\isasymexists}e{\isachardot}{\kern0pt}\ ForcePresentation\ e\ force{\isacharparenright}{\kern0pt}\ {\isasymlongrightarrow}\isanewline
\ \ \ \ \ \ \ {\isacharparenleft}{\kern0pt}{\isasymexists}e{\isachardot}{\kern0pt}\ Phenomenon\ e\ {\isasymand}\ Inseparable\ e\ {\isasymOmega}{\isacharparenright}{\kern0pt}{\isachardoublequoteclose}\isanewline
%
\isadelimproof
\ \ %
\endisadelimproof
%
\isatagproof
\isakeywordONE{proof}\isamarkupfalse%
\ {\isacharparenleft}{\kern0pt}intro\ allI\ impI{\isacharparenright}{\kern0pt}\isanewline
\ \ \ \ \isakeywordTHREE{fix}\isamarkupfalse%
\ force\isanewline
\ \ \ \ \isakeywordTHREE{assume}\isamarkupfalse%
\ {\isachardoublequoteopen}{\isasymexists}e{\isachardot}{\kern0pt}\ ForcePresentation\ e\ force{\isachardoublequoteclose}\isanewline
\ \ \ \ \isakeywordONE{then}\isamarkupfalse%
\ \isakeywordTHREE{obtain}\isamarkupfalse%
\ e\ \isakeywordTWO{where}\ FP{\isacharcolon}{\kern0pt}\ {\isachardoublequoteopen}ForcePresentation\ e\ force{\isachardoublequoteclose}\ \isakeywordONE{by}\isamarkupfalse%
\ blast\isanewline
\ \ \ \ \isakeywordONE{have}\isamarkupfalse%
\ {\isachardoublequoteopen}Phenomenon\ e{\isachardoublequoteclose}\ \isakeywordONE{using}\isamarkupfalse%
\ FP\ F{\isadigit{1}}{\isacharunderscore}{\kern0pt}forces{\isacharunderscore}{\kern0pt}phenomenal\ \isakeywordONE{by}\isamarkupfalse%
\ blast\isanewline
\ \ \ \ \isakeywordONE{moreover}\isamarkupfalse%
\ \isakeywordONE{have}\isamarkupfalse%
\ {\isachardoublequoteopen}Inseparable\ e\ {\isasymOmega}{\isachardoublequoteclose}\ \isakeywordONE{using}\isamarkupfalse%
\ FP\ Force{\isacharunderscore}{\kern0pt}phenomena{\isacharunderscore}{\kern0pt}nondual\ \isakeywordONE{by}\isamarkupfalse%
\ blast\isanewline
\ \ \ \ \isakeywordONE{ultimately}\isamarkupfalse%
\ \isakeywordTHREE{show}\isamarkupfalse%
\ {\isachardoublequoteopen}{\isasymexists}e{\isachardot}{\kern0pt}\ Phenomenon\ e\ {\isasymand}\ Inseparable\ e\ {\isasymOmega}{\isachardoublequoteclose}\ \isakeywordONE{by}\isamarkupfalse%
\ blast\isanewline
\ \ \isakeywordONE{qed}\isamarkupfalse%
%
\endisatagproof
{\isafoldproof}%
%
\isadelimproof
\isanewline
%
\endisadelimproof
\isanewline
\ \ \isakeywordONE{theorem}\isamarkupfalse%
\ Framework{\isacharunderscore}{\kern0pt}explains{\isacharunderscore}{\kern0pt}entanglement{\isacharcolon}{\kern0pt}\isanewline
\ \ \ \ {\isachardoublequoteopen}{\isasymforall}e{\isadigit{1}}\ e{\isadigit{2}}{\isachardot}{\kern0pt}\ Entangled\ e{\isadigit{1}}\ e{\isadigit{2}}\ {\isasymlongrightarrow}\isanewline
\ \ \ \ \ \ \ {\isacharparenleft}{\kern0pt}{\isasymexists}s{\isachardot}{\kern0pt}\ Substrate\ s\ {\isasymand}\ Presents\ e{\isadigit{1}}\ s\ {\isasymand}\ Presents\ e{\isadigit{2}}\ s\ {\isasymand}\ s\ {\isacharequal}{\kern0pt}\ {\isasymOmega}{\isacharparenright}{\kern0pt}{\isachardoublequoteclose}\isanewline
%
\isadelimproof
\ \ \ \ %
\endisadelimproof
%
\isatagproof
\isakeywordONE{using}\isamarkupfalse%
\ ENT{\isadigit{3}}{\isacharunderscore}{\kern0pt}substrate{\isacharunderscore}{\kern0pt}unity\ \isakeywordONE{by}\isamarkupfalse%
\ blast%
\endisatagproof
{\isafoldproof}%
%
\isadelimproof
\isanewline
%
\endisadelimproof
\isanewline
\ \ \isakeywordONE{theorem}\isamarkupfalse%
\ Framework{\isacharunderscore}{\kern0pt}supports{\isacharunderscore}{\kern0pt}gauge{\isacharunderscore}{\kern0pt}unification{\isacharcolon}{\kern0pt}\isanewline
\ \ \ \ {\isachardoublequoteopen}{\isasymforall}gg\ subgroups{\isachardot}{\kern0pt}\ Unified\ gg\ subgroups\ {\isasymlongrightarrow}\isanewline
\ \ \ \ \ \ \ {\isacharparenleft}{\kern0pt}{\isasymforall}sg\ {\isasymin}\ subgroups{\isachardot}{\kern0pt}\ {\isasymforall}e{\isachardot}{\kern0pt}\ IndexScheme\ sg\ e\ {\isasymlongrightarrow}\ IndexScheme\ gg\ e{\isacharparenright}{\kern0pt}{\isachardoublequoteclose}\isanewline
%
\isadelimproof
\ \ \ \ %
\endisadelimproof
%
\isatagproof
\isakeywordONE{using}\isamarkupfalse%
\ G{\isadigit{2}}{\isacharunderscore}{\kern0pt}unified{\isacharunderscore}{\kern0pt}preserves{\isacharunderscore}{\kern0pt}indexing\ \isakeywordONE{by}\isamarkupfalse%
\ blast%
\endisatagproof
{\isafoldproof}%
%
\isadelimproof
%
\endisadelimproof
%
\isadelimdocument
%
\endisadelimdocument
%
\isatagdocument
%
\isamarkupsection{Final Integration%
}
\isamarkuptrue%
%
\endisatagdocument
{\isafolddocument}%
%
\isadelimdocument
%
\endisadelimdocument
\isakeywordONE{theorem}\isamarkupfalse%
\ Complete{\isacharunderscore}{\kern0pt}Unification{\isacharcolon}{\kern0pt}\isanewline
\ \ {\isachardoublequoteopen}{\isacharparenleft}{\kern0pt}{\isasymforall}e{\isachardot}{\kern0pt}\ Phenomenon\ e\ {\isasymlongrightarrow}\ Inseparable\ e\ {\isasymOmega}{\isacharparenright}{\kern0pt}\ {\isasymand}\isanewline
\ \ \ {\isacharparenleft}{\kern0pt}{\isasymforall}e{\isadigit{1}}\ e{\isadigit{2}}{\isachardot}{\kern0pt}\ Phenomenon\ e{\isadigit{1}}\ {\isasymand}\ Phenomenon\ e{\isadigit{2}}\ {\isasymlongrightarrow}\isanewline
\ \ \ \ \ \ {\isacharparenleft}{\kern0pt}{\isasymexists}s{\isachardot}{\kern0pt}\ Substrate\ s\ {\isasymand}\ Presents\ e{\isadigit{1}}\ s\ {\isasymand}\ Presents\ e{\isadigit{2}}\ s{\isacharparenright}{\kern0pt}{\isacharparenright}{\kern0pt}\ {\isasymand}\isanewline
\ \ \ {\isacharparenleft}{\kern0pt}{\isasymforall}force\ e{\isachardot}{\kern0pt}\ ForcePresentation\ e\ force\ {\isasymlongrightarrow}\ Presents\ e\ {\isasymOmega}{\isacharparenright}{\kern0pt}\ {\isasymand}\isanewline
\ \ \ {\isacharparenleft}{\kern0pt}{\isasymforall}e{\isadigit{1}}\ e{\isadigit{2}}{\isachardot}{\kern0pt}\ Entangled\ e{\isadigit{1}}\ e{\isadigit{2}}\ {\isasymlongrightarrow}\ Presents\ e{\isadigit{1}}\ {\isasymOmega}\ {\isasymand}\ Presents\ e{\isadigit{2}}\ {\isasymOmega}{\isacharparenright}{\kern0pt}\ {\isasymand}\isanewline
\ \ \ {\isacharparenleft}{\kern0pt}{\isasymforall}ft\ e{\isachardot}{\kern0pt}\ Excitation\ e\ ft\ {\isasymlongrightarrow}\ Presents\ e\ {\isasymOmega}{\isacharparenright}{\kern0pt}{\isachardoublequoteclose}\isanewline
%
\isadelimproof
%
\endisadelimproof
%
\isatagproof
\isakeywordONE{proof}\isamarkupfalse%
\ {\isacharparenleft}{\kern0pt}intro\ conjI{\isacharparenright}{\kern0pt}\isanewline
\ \ \isakeywordTHREE{show}\isamarkupfalse%
\ {\isachardoublequoteopen}{\isasymforall}e{\isachardot}{\kern0pt}\ Phenomenon\ e\ {\isasymlongrightarrow}\ Inseparable\ e\ {\isasymOmega}{\isachardoublequoteclose}\isanewline
\ \ \ \ \isakeywordONE{using}\isamarkupfalse%
\ Nonduality\ \isakeywordONE{by}\isamarkupfalse%
\ blast\isanewline
\isakeywordONE{next}\isamarkupfalse%
\isanewline
\ \ \isakeywordTHREE{show}\isamarkupfalse%
\ {\isachardoublequoteopen}{\isasymforall}e{\isadigit{1}}\ e{\isadigit{2}}{\isachardot}{\kern0pt}\ Phenomenon\ e{\isadigit{1}}\ {\isasymand}\ Phenomenon\ e{\isadigit{2}}\ {\isasymlongrightarrow}\isanewline
\ \ \ \ \ \ \ \ {\isacharparenleft}{\kern0pt}{\isasymexists}s{\isachardot}{\kern0pt}\ Substrate\ s\ {\isasymand}\ Presents\ e{\isadigit{1}}\ s\ {\isasymand}\ Presents\ e{\isadigit{2}}\ s{\isacharparenright}{\kern0pt}{\isachardoublequoteclose}\isanewline
\ \ \ \ \isakeywordONE{using}\isamarkupfalse%
\ substrate{\isacharunderscore}{\kern0pt}Omega\ A{\isadigit{4}}{\isacharunderscore}{\kern0pt}presentation\ \isakeywordONE{by}\isamarkupfalse%
\ blast\isanewline
\isakeywordONE{next}\isamarkupfalse%
\isanewline
\ \ \isakeywordTHREE{show}\isamarkupfalse%
\ {\isachardoublequoteopen}{\isasymforall}force\ e{\isachardot}{\kern0pt}\ ForcePresentation\ e\ force\ {\isasymlongrightarrow}\ Presents\ e\ {\isasymOmega}{\isachardoublequoteclose}\isanewline
\ \ \ \ \isakeywordONE{using}\isamarkupfalse%
\ F{\isadigit{3}}{\isacharunderscore}{\kern0pt}forces{\isacharunderscore}{\kern0pt}via{\isacharunderscore}{\kern0pt}presentation\ \isakeywordONE{by}\isamarkupfalse%
\ blast\isanewline
\isakeywordONE{next}\isamarkupfalse%
\isanewline
\ \ \isakeywordTHREE{show}\isamarkupfalse%
\ {\isachardoublequoteopen}{\isasymforall}e{\isadigit{1}}\ e{\isadigit{2}}{\isachardot}{\kern0pt}\ Entangled\ e{\isadigit{1}}\ e{\isadigit{2}}\ {\isasymlongrightarrow}\ Presents\ e{\isadigit{1}}\ {\isasymOmega}\ {\isasymand}\ Presents\ e{\isadigit{2}}\ {\isasymOmega}{\isachardoublequoteclose}\isanewline
\ \ \ \ \isakeywordONE{using}\isamarkupfalse%
\ ENT{\isadigit{3}}{\isacharunderscore}{\kern0pt}substrate{\isacharunderscore}{\kern0pt}unity\ \isakeywordONE{by}\isamarkupfalse%
\ blast\isanewline
\isakeywordONE{next}\isamarkupfalse%
\isanewline
\ \ \isakeywordTHREE{show}\isamarkupfalse%
\ {\isachardoublequoteopen}{\isasymforall}ft\ e{\isachardot}{\kern0pt}\ Excitation\ e\ ft\ {\isasymlongrightarrow}\ Presents\ e\ {\isasymOmega}{\isachardoublequoteclose}\isanewline
\ \ \ \ \isakeywordONE{using}\isamarkupfalse%
\ FC{\isadigit{2}}{\isacharunderscore}{\kern0pt}excitations{\isacharunderscore}{\kern0pt}are{\isacharunderscore}{\kern0pt}phenomena\ substrate{\isacharunderscore}{\kern0pt}Omega\ A{\isadigit{4}}{\isacharunderscore}{\kern0pt}presentation\ \isakeywordONE{by}\isamarkupfalse%
\ blast\isanewline
\isakeywordONE{qed}\isamarkupfalse%
%
\endisatagproof
{\isafoldproof}%
%
\isadelimproof
%
\endisadelimproof
%
\begin{isamarkuptext}%
The Complete_Unification theorem establishes that:
  1. All phenomena are inseparable from the unique substrate.
  2. Any two phenomena share (present) the same substrate.
  3. All force presentations are presentations of the substrate.
  4. All entangled phenomena are presentations of the substrate.
  5. All field excitations are presentations of the substrate.

  This proves ontological unification: the apparent diversity of fields, forces,
  and particles reflects different presentation modes of the singular substrate,
  not fundamental ontological plurality.%
\end{isamarkuptext}\isamarkuptrue%
%
\isadelimdocument
%
\endisadelimdocument
%
\isatagdocument
%
\isamarkupsection{Nitpick Sanity Checks (Safe, Non-intrusive)%
}
\isamarkuptrue%
%
\endisatagdocument
{\isafolddocument}%
%
\isadelimdocument
%
\endisadelimdocument
\ \ \isakeywordONE{lemma}\isamarkupfalse%
\ nitpick{\isacharunderscore}{\kern0pt}sanity{\isacharunderscore}{\kern0pt}axioms{\isacharunderscore}{\kern0pt}satisfiable{\isacharcolon}{\kern0pt}\ True\isanewline
\ \ \ \ \isakeywordONE{nitpick}\isamarkupfalse%
\ {\isacharbrackleft}{\kern0pt}satisfy{\isacharcomma}{\kern0pt}\ card\ {\isacharequal}{\kern0pt}\ {\isadigit{2}}{\isacharcomma}{\kern0pt}{\isadigit{3}}{\isacharbrackright}{\kern0pt}\ \ \isanewline
%
\isadelimproof
\ \ \ \ %
\endisadelimproof
%
\isatagproof
\isakeywordONE{by}\isamarkupfalse%
\ simp%
\endisatagproof
{\isafoldproof}%
%
\isadelimproof
%
\endisadelimproof
\isanewline
\isanewline
\ \ \isanewline
\ \ \isakeywordONE{lemma}\isamarkupfalse%
\ nitpick{\isacharunderscore}{\kern0pt}reg{\isacharunderscore}{\kern0pt}nonduality{\isacharunderscore}{\kern0pt}invariant{\isacharcolon}{\kern0pt}\isanewline
\ \ \ \ {\isachardoublequoteopen}{\isasymforall}e{\isachardot}{\kern0pt}\ Phenomenon\ e\ {\isasymlongrightarrow}\ Inseparable\ e\ {\isasymOmega}{\isachardoublequoteclose}\isanewline
\ \ \ \ \isakeywordONE{nitpick}\isamarkupfalse%
\ {\isacharbrackleft}{\kern0pt}card\ {\isacharequal}{\kern0pt}\ {\isadigit{2}}{\isacharbrackright}{\kern0pt}\ \ \isanewline
%
\isadelimproof
\ \ \ \ %
\endisadelimproof
%
\isatagproof
\isakeywordONE{by}\isamarkupfalse%
\ {\isacharparenleft}{\kern0pt}simp\ add{\isacharcolon}{\kern0pt}\ Nonduality{\isacharparenright}{\kern0pt}%
\endisatagproof
{\isafoldproof}%
%
\isadelimproof
%
\endisadelimproof
\isanewline
\isanewline
\ \ \isanewline
\ \ \isakeywordONE{lemma}\isamarkupfalse%
\ nitpick{\isacharunderscore}{\kern0pt}reg{\isacharunderscore}{\kern0pt}entanglement{\isacharunderscore}{\kern0pt}has{\isacharunderscore}{\kern0pt}substrate{\isacharcolon}{\kern0pt}\isanewline
\ \ \ \ {\isachardoublequoteopen}{\isasymforall}e{\isadigit{1}}\ e{\isadigit{2}}{\isachardot}{\kern0pt}\ Entangled\ e{\isadigit{1}}\ e{\isadigit{2}}\ {\isasymlongrightarrow}\isanewline
\ \ \ \ \ \ \ \ {\isacharparenleft}{\kern0pt}{\isasymexists}s{\isachardot}{\kern0pt}\ Substrate\ s\ {\isasymand}\ Presents\ e{\isadigit{1}}\ s\ {\isasymand}\ Presents\ e{\isadigit{2}}\ s\ {\isasymand}\ s\ {\isacharequal}{\kern0pt}\ {\isasymOmega}{\isacharparenright}{\kern0pt}{\isachardoublequoteclose}\isanewline
\ \ \ \ \isakeywordONE{nitpick}\isamarkupfalse%
\ {\isacharbrackleft}{\kern0pt}card\ {\isacharequal}{\kern0pt}\ {\isadigit{2}}{\isacharbrackright}{\kern0pt}\ \ \isanewline
%
\isadelimproof
\ \ \ \ %
\endisadelimproof
%
\isatagproof
\isakeywordONE{by}\isamarkupfalse%
\ {\isacharparenleft}{\kern0pt}meson\ ENT{\isadigit{3}}{\isacharunderscore}{\kern0pt}substrate{\isacharunderscore}{\kern0pt}unity{\isacharparenright}{\kern0pt}%
\endisatagproof
{\isafoldproof}%
%
\isadelimproof
%
\endisadelimproof
\isanewline
%
\isadelimtheory
\isanewline
%
\endisadelimtheory
%
\isatagtheory
\isakeywordTWO{end}\isamarkupfalse%
%
\endisatagtheory
{\isafoldtheory}%
%
\isadelimtheory
%
\endisadelimtheory
%
\end{isabellebody}%
\endinput
%:%file=~/Documents/GitHub/unified_Field_theory/Unified_Field_Theory/Unified_Field_Theory.thy%:%
%:%10=1%:%
%:%11=1%:%
%:%12=2%:%
%:%13=3%:%
%:%18=3%:%
%:%21=4%:%
%:%22=5%:%
%:%23=6%:%
%:%24=7%:%
%:%25=8%:%
%:%26=9%:%
%:%27=10%:%
%:%28=11%:%
%:%29=12%:%
%:%30=13%:%
%:%31=14%:%
%:%32=15%:%
%:%33=16%:%
%:%34=16%:%
%:%41=18%:%
%:%51=20%:%
%:%52=20%:%
%:%53=21%:%
%:%54=22%:%
%:%55=22%:%
%:%56=23%:%
%:%57=24%:%
%:%58=25%:%
%:%59=26%:%
%:%60=27%:%
%:%61=27%:%
%:%62=28%:%
%:%63=29%:%
%:%64=30%:%
%:%65=31%:%
%:%66=32%:%
%:%67=33%:%
%:%68=33%:%
%:%69=34%:%
%:%70=35%:%
%:%73=36%:%
%:%77=36%:%
%:%78=36%:%
%:%79=36%:%
%:%84=36%:%
%:%87=37%:%
%:%88=38%:%
%:%89=38%:%
%:%90=39%:%
%:%91=40%:%
%:%94=41%:%
%:%98=41%:%
%:%99=41%:%
%:%100=41%:%
%:%114=44%:%
%:%124=46%:%
%:%125=46%:%
%:%126=47%:%
%:%127=48%:%
%:%128=48%:%
%:%129=49%:%
%:%130=50%:%
%:%131=51%:%
%:%132=52%:%
%:%133=53%:%
%:%134=53%:%
%:%135=54%:%
%:%136=55%:%
%:%137=56%:%
%:%138=57%:%
%:%139=58%:%
%:%140=59%:%
%:%141=60%:%
%:%142=61%:%
%:%143=62%:%
%:%144=62%:%
%:%145=63%:%
%:%146=64%:%
%:%149=65%:%
%:%153=65%:%
%:%154=65%:%
%:%155=65%:%
%:%160=65%:%
%:%163=66%:%
%:%164=67%:%
%:%165=67%:%
%:%166=68%:%
%:%167=69%:%
%:%170=70%:%
%:%174=70%:%
%:%175=70%:%
%:%176=70%:%
%:%190=73%:%
%:%200=75%:%
%:%201=75%:%
%:%202=76%:%
%:%203=77%:%
%:%204=77%:%
%:%205=78%:%
%:%206=79%:%
%:%207=80%:%
%:%208=81%:%
%:%209=81%:%
%:%210=82%:%
%:%211=83%:%
%:%212=84%:%
%:%213=85%:%
%:%215=87%:%
%:%216=88%:%
%:%217=89%:%
%:%218=90%:%
%:%219=91%:%
%:%220=91%:%
%:%221=92%:%
%:%222=93%:%
%:%225=94%:%
%:%229=94%:%
%:%230=94%:%
%:%231=94%:%
%:%236=94%:%
%:%239=95%:%
%:%240=96%:%
%:%241=96%:%
%:%242=97%:%
%:%243=98%:%
%:%246=99%:%
%:%250=99%:%
%:%251=99%:%
%:%252=100%:%
%:%253=100%:%
%:%254=101%:%
%:%255=101%:%
%:%256=102%:%
%:%257=102%:%
%:%258=102%:%
%:%259=102%:%
%:%260=103%:%
%:%261=103%:%
%:%262=104%:%
%:%263=104%:%
%:%264=104%:%
%:%265=105%:%
%:%280=108%:%
%:%290=110%:%
%:%291=110%:%
%:%292=111%:%
%:%293=112%:%
%:%294=113%:%
%:%295=114%:%
%:%296=114%:%
%:%297=115%:%
%:%298=116%:%
%:%299=117%:%
%:%300=118%:%
%:%301=119%:%
%:%302=120%:%
%:%303=121%:%
%:%304=122%:%
%:%305=123%:%
%:%306=124%:%
%:%307=125%:%
%:%308=125%:%
%:%309=126%:%
%:%312=127%:%
%:%316=127%:%
%:%317=127%:%
%:%318=128%:%
%:%319=128%:%
%:%320=129%:%
%:%321=129%:%
%:%322=130%:%
%:%323=130%:%
%:%324=130%:%
%:%325=130%:%
%:%326=131%:%
%:%327=131%:%
%:%328=131%:%
%:%329=131%:%
%:%330=132%:%
%:%336=132%:%
%:%339=133%:%
%:%340=134%:%
%:%341=134%:%
%:%342=135%:%
%:%343=136%:%
%:%344=137%:%
%:%345=138%:%
%:%348=139%:%
%:%352=139%:%
%:%353=139%:%
%:%354=139%:%
%:%368=142%:%
%:%378=144%:%
%:%379=144%:%
%:%380=145%:%
%:%381=146%:%
%:%382=146%:%
%:%383=147%:%
%:%384=148%:%
%:%385=149%:%
%:%386=150%:%
%:%387=151%:%
%:%388=152%:%
%:%389=153%:%
%:%390=153%:%
%:%391=154%:%
%:%392=155%:%
%:%393=156%:%
%:%394=157%:%
%:%395=158%:%
%:%396=159%:%
%:%397=160%:%
%:%398=161%:%
%:%399=161%:%
%:%400=162%:%
%:%401=163%:%
%:%404=164%:%
%:%408=164%:%
%:%409=164%:%
%:%410=164%:%
%:%415=164%:%
%:%418=165%:%
%:%419=166%:%
%:%420=166%:%
%:%421=167%:%
%:%424=168%:%
%:%428=168%:%
%:%429=168%:%
%:%430=168%:%
%:%444=171%:%
%:%454=173%:%
%:%455=173%:%
%:%456=174%:%
%:%457=175%:%
%:%458=176%:%
%:%459=177%:%
%:%460=177%:%
%:%461=178%:%
%:%462=179%:%
%:%463=180%:%
%:%464=181%:%
%:%465=182%:%
%:%466=183%:%
%:%467=184%:%
%:%468=185%:%
%:%469=186%:%
%:%470=187%:%
%:%471=187%:%
%:%472=188%:%
%:%473=189%:%
%:%476=190%:%
%:%480=190%:%
%:%481=190%:%
%:%482=190%:%
%:%487=190%:%
%:%490=191%:%
%:%491=192%:%
%:%492=193%:%
%:%493=193%:%
%:%494=194%:%
%:%497=195%:%
%:%501=195%:%
%:%502=195%:%
%:%503=196%:%
%:%504=196%:%
%:%505=197%:%
%:%506=197%:%
%:%507=198%:%
%:%508=198%:%
%:%509=198%:%
%:%510=199%:%
%:%511=199%:%
%:%512=199%:%
%:%513=200%:%
%:%514=200%:%
%:%515=200%:%
%:%516=201%:%
%:%517=201%:%
%:%518=201%:%
%:%519=202%:%
%:%520=202%:%
%:%521=202%:%
%:%522=202%:%
%:%523=203%:%
%:%538=206%:%
%:%548=208%:%
%:%549=208%:%
%:%550=209%:%
%:%551=210%:%
%:%552=211%:%
%:%553=212%:%
%:%554=212%:%
%:%555=213%:%
%:%556=214%:%
%:%557=215%:%
%:%558=216%:%
%:%559=217%:%
%:%560=218%:%
%:%561=219%:%
%:%562=220%:%
%:%563=221%:%
%:%564=221%:%
%:%565=222%:%
%:%566=223%:%
%:%569=224%:%
%:%573=224%:%
%:%574=224%:%
%:%575=224%:%
%:%580=224%:%
%:%583=225%:%
%:%584=226%:%
%:%585=226%:%
%:%586=227%:%
%:%589=228%:%
%:%593=228%:%
%:%594=228%:%
%:%595=229%:%
%:%596=229%:%
%:%597=230%:%
%:%598=230%:%
%:%599=231%:%
%:%600=231%:%
%:%601=232%:%
%:%602=232%:%
%:%603=232%:%
%:%604=233%:%
%:%605=233%:%
%:%606=233%:%
%:%607=233%:%
%:%608=234%:%
%:%623=237%:%
%:%633=239%:%
%:%634=239%:%
%:%635=240%:%
%:%640=245%:%
%:%643=246%:%
%:%647=246%:%
%:%648=246%:%
%:%649=247%:%
%:%650=247%:%
%:%651=248%:%
%:%652=248%:%
%:%653=249%:%
%:%654=249%:%
%:%657=252%:%
%:%658=253%:%
%:%659=253%:%
%:%660=254%:%
%:%661=254%:%
%:%662=255%:%
%:%663=255%:%
%:%664=255%:%
%:%665=255%:%
%:%666=256%:%
%:%667=256%:%
%:%668=257%:%
%:%669=257%:%
%:%670=258%:%
%:%671=258%:%
%:%672=258%:%
%:%673=258%:%
%:%674=259%:%
%:%675=259%:%
%:%676=260%:%
%:%677=260%:%
%:%678=261%:%
%:%679=261%:%
%:%680=261%:%
%:%681=261%:%
%:%682=262%:%
%:%683=262%:%
%:%684=263%:%
%:%685=263%:%
%:%686=264%:%
%:%687=264%:%
%:%688=265%:%
%:%689=265%:%
%:%690=265%:%
%:%691=265%:%
%:%692=265%:%
%:%693=266%:%
%:%694=266%:%
%:%695=267%:%
%:%696=267%:%
%:%697=268%:%
%:%698=268%:%
%:%699=269%:%
%:%700=269%:%
%:%701=269%:%
%:%702=269%:%
%:%703=269%:%
%:%704=270%:%
%:%705=270%:%
%:%706=271%:%
%:%712=271%:%
%:%715=272%:%
%:%716=273%:%
%:%717=273%:%
%:%718=274%:%
%:%719=275%:%
%:%722=276%:%
%:%726=276%:%
%:%727=276%:%
%:%728=276%:%
%:%733=276%:%
%:%736=277%:%
%:%737=278%:%
%:%738=278%:%
%:%739=279%:%
%:%740=280%:%
%:%743=281%:%
%:%747=281%:%
%:%748=281%:%
%:%749=281%:%
%:%763=284%:%
%:%773=286%:%
%:%774=286%:%
%:%775=287%:%
%:%776=288%:%
%:%777=289%:%
%:%778=290%:%
%:%779=290%:%
%:%780=291%:%
%:%781=292%:%
%:%782=293%:%
%:%783=294%:%
%:%784=295%:%
%:%785=296%:%
%:%786=297%:%
%:%787=298%:%
%:%788=299%:%
%:%789=299%:%
%:%790=300%:%
%:%791=301%:%
%:%794=302%:%
%:%798=302%:%
%:%799=302%:%
%:%800=303%:%
%:%801=303%:%
%:%802=304%:%
%:%803=304%:%
%:%804=304%:%
%:%805=305%:%
%:%806=305%:%
%:%807=306%:%
%:%808=306%:%
%:%809=307%:%
%:%810=307%:%
%:%811=307%:%
%:%812=308%:%
%:%818=308%:%
%:%821=309%:%
%:%822=310%:%
%:%823=310%:%
%:%824=311%:%
%:%826=313%:%
%:%829=314%:%
%:%833=314%:%
%:%834=314%:%
%:%835=314%:%
%:%849=317%:%
%:%859=321%:%
%:%860=321%:%
%:%861=322%:%
%:%862=323%:%
%:%863=324%:%
%:%864=325%:%
%:%865=326%:%
%:%866=327%:%
%:%867=328%:%
%:%868=329%:%
%:%869=330%:%
%:%870=331%:%
%:%871=332%:%
%:%872=332%:%
%:%873=333%:%
%:%874=334%:%
%:%875=335%:%
%:%876=336%:%
%:%877=337%:%
%:%878=338%:%
%:%879=339%:%
%:%880=340%:%
%:%881=340%:%
%:%882=341%:%
%:%883=342%:%
%:%886=343%:%
%:%890=343%:%
%:%891=343%:%
%:%892=343%:%
%:%906=346%:%
%:%916=349%:%
%:%917=349%:%
%:%919=349%:%
%:%923=349%:%
%:%924=349%:%
%:%931=349%:%
%:%932=350%:%
%:%933=351%:%
%:%934=351%:%
%:%935=352%:%
%:%936=353%:%
%:%939=354%:%
%:%943=354%:%
%:%944=354%:%
%:%945=355%:%
%:%946=355%:%
%:%947=356%:%
%:%948=356%:%
%:%949=357%:%
%:%950=357%:%
%:%951=357%:%
%:%952=357%:%
%:%953=358%:%
%:%954=358%:%
%:%955=358%:%
%:%956=358%:%
%:%957=359%:%
%:%958=359%:%
%:%959=359%:%
%:%960=359%:%
%:%961=359%:%
%:%962=360%:%
%:%963=360%:%
%:%964=360%:%
%:%965=360%:%
%:%966=361%:%
%:%972=361%:%
%:%975=362%:%
%:%976=363%:%
%:%977=363%:%
%:%978=364%:%
%:%979=365%:%
%:%982=366%:%
%:%986=366%:%
%:%987=366%:%
%:%988=366%:%
%:%993=366%:%
%:%996=367%:%
%:%997=368%:%
%:%998=368%:%
%:%999=369%:%
%:%1000=370%:%
%:%1003=371%:%
%:%1007=371%:%
%:%1008=371%:%
%:%1009=371%:%
%:%1023=374%:%
%:%1033=379%:%
%:%1034=379%:%
%:%1035=380%:%
%:%1040=385%:%
%:%1047=386%:%
%:%1048=386%:%
%:%1049=387%:%
%:%1050=387%:%
%:%1051=388%:%
%:%1052=388%:%
%:%1053=388%:%
%:%1054=389%:%
%:%1055=389%:%
%:%1056=390%:%
%:%1057=390%:%
%:%1058=391%:%
%:%1059=392%:%
%:%1060=392%:%
%:%1061=392%:%
%:%1062=393%:%
%:%1063=393%:%
%:%1064=394%:%
%:%1065=394%:%
%:%1066=395%:%
%:%1067=395%:%
%:%1068=395%:%
%:%1069=396%:%
%:%1070=396%:%
%:%1071=397%:%
%:%1072=397%:%
%:%1073=398%:%
%:%1074=398%:%
%:%1075=398%:%
%:%1076=399%:%
%:%1077=399%:%
%:%1078=400%:%
%:%1079=400%:%
%:%1080=401%:%
%:%1081=401%:%
%:%1082=401%:%
%:%1083=402%:%
%:%1093=406%:%
%:%1094=407%:%
%:%1095=408%:%
%:%1096=409%:%
%:%1097=410%:%
%:%1098=411%:%
%:%1099=412%:%
%:%1100=413%:%
%:%1101=414%:%
%:%1102=415%:%
%:%1111=419%:%
%:%1121=422%:%
%:%1122=422%:%
%:%1123=423%:%
%:%1124=423%:%
%:%1127=424%:%
%:%1131=424%:%
%:%1132=424%:%
%:%1139=424%:%
%:%1140=425%:%
%:%1141=426%:%
%:%1142=427%:%
%:%1143=427%:%
%:%1144=428%:%
%:%1145=429%:%
%:%1146=429%:%
%:%1149=430%:%
%:%1153=430%:%
%:%1154=430%:%
%:%1161=430%:%
%:%1162=431%:%
%:%1163=432%:%
%:%1164=433%:%
%:%1165=433%:%
%:%1166=434%:%
%:%1167=435%:%
%:%1168=436%:%
%:%1169=436%:%
%:%1172=437%:%
%:%1176=437%:%
%:%1177=437%:%
%:%1184=437%:%
%:%1187=438%:%
%:%1192=439%:%



\bibliographystyle{abbrv}
\bibliography{root}
\end{document}
